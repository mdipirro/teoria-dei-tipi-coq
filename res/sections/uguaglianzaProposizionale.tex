\newpage
\setcounter{section}{4}
\section{Tipo dell'uguaglianza proposizionale alla Gentzen/Lawvere e alla Leibniz}
\subsection{Esercizio 1}
\begin{thm}
	In che relazione stanno i due tipi di uguaglianza?
\end{thm}
L'uguaglianza a là Gentzen/Lawvere (utilizzata nel calcolo dei sequenti) è meno potente di quella a là Leibniz. Si nota, infatti, che nella regola di eliminazione della prima il termine $c~\in~C(a,a)~[\Gamma]$ non è dipendente da nessun elemento di A. Nella seconda, al contrario, questo termine è dipendete da un $x~\in~A$ dato nel contesto. Di conseguenza, con la prima la scelta del termine $a~\in~A~[\Gamma]$ vincola anche il termine $c~\in~C(a,a)~[\Gamma]$, cosa che non accade con la seconda, nella quale $x~\in~A$ è una variabile inserita nel contesto.

\subsection{Esercizio 2}
\begin{thm}
	Con quale uguaglianza è possibile derivare un proof-term $q\in Id(B, f(x), f(y)) [x\in A, y\in A, w\in Id(A, x, y)]$, dove $f(x)\in B[x\in A]$?
\end{thm}
Può essere derivato con entrambe le uguaglianze, infatti:

\lstinputlisting[language=Coq]{res/code/uguaglianzaProposizionale/es2corretto.v}

\subsection{Esercizio 3}
\begin{thm} Trovare un proof-term
	\[ \mathbf{pf}\in P(b)\ [a\in A, b\in A, z\in P(a), w\in \Id{A}{a}{b}]\]
\end{thm}
Può essere derivato con entrambe le uguaglianze, infatti:

\lstinputlisting[language=Coq]{res/code/uguaglianzaProposizionale/es3corretto.v}

\subsection{Esercizio 4}
\begin{thm}
	Trovare un proof-term
	\[ \mathbf{pf} \in \Id{A}{a}{c}\ [a\in A, b\in A, c\in A, w_1 \in \Id{A}{a}{b}, w_2 \in \Id{A}{b}{c} ] \]
\end{thm}

Può essere derivato con entrambe le uguaglianze, infatti:

\lstinputlisting[language=Coq]{res/code/uguaglianzaProposizionale/es4corretto.v}

\subsection{Esercizio 5}
\begin{thm}
	Con quale uguaglianza è derivabile il proof-term $pf\in Id(N_1, x, \ast)~[x\in N_1]$?
\end{thm}
Si deriva con entrambe le uguaglianze, infatti:
\lstinputlisting[language=Coq]{res/code/uguaglianzaProposizionale/es5corretto.v}

\subsection{Esercizio 6}
\begin{thm}
	Con quale uguaglianza è possibile definire l'addizione tra naturali e derivare i due proof-terms $pf_1\in Id(nat, x+0,x)~[x\in Nat]$ e $pf_2\in Id(nat, 0+x,x)~[x\in Nat]$?
\end{thm}
Con entrambe le uguaglianze, infatti:
\lstinputlisting[language=Coq]{res/code/uguaglianzaProposizionale/es6corretto.v}

\subsection{Esercizio 7}
\begin{thm}
	Quali assiomi dell'aritmetica di Peano sono derivabili con i tipi descritti finora?
\end{thm}
\lstinputlisting[language=Coq]{res/code/uguaglianzaProposizionale/es7.v}

\subsection{Esercizio 8}
\begin{thm}
	Si dimostri che esiste un proof-term \textbf{pf} tale che
	\[pf~\in~Id(Nat, x~+_1~y, x~+_2~y)~[x~\in~Nat,~y~\in~Nat]\]
\end{thm}
\subsubsection{Commutatività della somma a mano}
Trovare un proof-term del tipo
\[ \mathrm{comm}(a, b) \in \Id{Nat}{a+b}{b+a}\ [a\in Nat,\ b\in Nat] \]

Definiamo il proof-term per induzione sull'addendo $b$.\\
Posto $\Gamma \equiv a\in Nat,\ b\in Nat$

\begin{scriptsize}
	\[ \infer{\mathrm{comm}(a, b) \equiv \mathrm{El}_{Nat}(b, d, (x, y).e(x,y)) \in \Id{Nat}{a+b}{b+a}\ [\Gamma]}{\begin{array}{ll} b\in Nat\ [\Gamma] & \Id{Nat}{a+x}{x+a}\ type\ [\Gamma,\ x\in Nat] \\ d\in \Id{Nat}{a+0}{0+a}\ [\Gamma] & e(x, y)\in \Id{Nat}{a+\suc{x}}{\suc{x}+a}\ [\Gamma, x\in Nat, y\in \Id{Nat}{a+x}{x+a}]\end{array}} \]
\end{scriptsize}

Occorre quindi definire 
\[ d\in \Id{Nat}{a+0}{0+a}\ [\Gamma] \]
Notiamo intanto che

\[ \infer[conv)]{d\in \Id{Nat}{a+0}{0+a}\ [\Gamma]}
{	d\in \Id{Nat}{a}{0+a}\ [\Gamma] & 
	\infer[sym)]{\Id{Nat}{a}{0+a} = \Id{Nat}{a+0}{0+a}\ type\ [\Gamma]}
	{
		\infer[sub-eq-typ)]{\Id{Nat}{a+0}{0+a}=\Id{Nat}{a}{0+a}\ type\ [\Gamma]}
		{a+0 = a\in Nat\ [\Gamma] & \Id{Nat}{z}{0+a}\ type\ [\Gamma, z\in Nat]}
	}
}
\]
per cui è sufficiente dimostrare che esiste
\[ d\in \Id{Nat}{a}{0+a}\ [\Gamma]\]
Procediamo per induzione su $a\in Nat$.

\begin{scriptsize}
	\[ \infer{d \equiv \mathrm{El}_{Nat}(a, d_1, (x, y).e_1(x,y)) \in \Id{Nat}{a}{0+a}\ [\Gamma]}{\begin{array}{ll} a\in Nat\ [\Gamma] & \Id{Nat}{x}{0+x}\ type\ [\Gamma,\ x\in Nat] \\ d_1\in \Id{Nat}{0}{0+0}\ [\Gamma] & e_1(x, y)\in \Id{Nat}{\suc{x}}{0+\suc{x}}\ [\Gamma, x\in Nat, y\in \Id{Nat}{x}{0+x}]\end{array}} \]
\end{scriptsize}
dove $d_1 \equiv \mathrm{id}(0)$, infatti
\[ \infer[conv)]{\mathrm{id}(0)\in \Id{Nat}{0}{0+0}\ [\Gamma]}
{	\mathrm{id}(0)\in \Id{Nat}{0}{0}\ [\Gamma] & 
	\infer[sym)]{\Id{Nat}{0}{0} = \Id{Nat}{0}{0+0}\ type\ [\Gamma]}
	{
		\infer[sub-eq-typ)]{\Id{Nat}{0}{0+0}=\Id{Nat}{0}{0}\ type\ [\Gamma]}
		{0+0 = 0\in Nat\ [\Gamma] & \Id{Nat}{0}{z}\ type\ [\Gamma, z\in Nat]}
	}
}
\]
mentre sappiamo che
\[ \mathrm{El}_{Id}(y, (x'').\mathrm{id}(\suc{x''})) \in \Id{Nat}{\suc{x}}{\suc{0+x}}\]
e per conversione, dato che $0+\suc{x} = \suc{0+x}$, prendiamo
\[ e_1(x, y) \equiv \mathrm{El}_{Id}(y, (x'').\mathrm{id}(\suc{x''})) \]
pertanto
\[d \equiv \mathrm{El}_{Nat}(a, \mathrm{id}(0), (x, y).\mathrm{El}_{Id}(y, (x'').\mathrm{id}(\suc{x''}))) \in \Id{Nat}{a}{0+a}\ [\Gamma]\]

Mostriamo adesso l'esistenza del proof-term 

\[e(x, y)\in \Id{Nat}{a+\suc{x}}{\suc{x}+a}\ [\Gamma, x\in Nat, y\in \Id{Nat}{a+x}{x+a}]\]

Dato che $a+\suc{x} = \suc{a+x}$, applicando $sub-eq-typ),\ sym),\ conv)$ ci basta trovare un proof-term

\[e(x, y)\in \Id{Nat}{\suc{a+x}}{\suc{x}+a}\ [\Gamma, x\in Nat, y\in \Id{Nat}{a+x}{x+a}]\]

L'ipotesi $y \in \Id{Nat}{a+x}{x+a}$ ci consente di avere il proof-term
\[ \mathrm{El}_{\mathrm{Id}}(y, (x'').\mathrm{id}(\suc{x''})) \in \Id{Nat}{\suc{a+x}}{\suc{x+a}}\]

Per cui, se dimostrassimo che esiste $w \in \Id{Nat}{\suc{x+a}}{\suc{x}+a}$ potremmo concludere per la transitività dell'uguaglianza di Leibniz. Tale proof-term si può ottenere per induzione su $a\in Nat$.

Posto $\Delta = \Gamma, x\in Nat, y\in \Id{Nat}{a+x}{x+a}$

\begin{scriptsize}
	\[ \infer{w \equiv \mathrm{El}_{Nat}(a, w_0, (x, y).w_1(x,y)) \in \Id{Nat}{\suc{x+a}}{\suc{x}+a}\ [\Delta]}{\begin{array}{l}
		\Id{Nat}{\suc{z+x}}{\suc{x}+z}\ type\ [\Delta] \\ a\in Nat\ [\Delta] \\  w_0\in \Id{Nat}{\suc{x+0}}{\suc{x}+0}\ [\Delta] \\ w_1(z, r)\in \Id{Nat}{\suc{x+\suc{z}}}{\suc{x}+\suc{z}}\ [\Delta z\in Nat, r\in \Id{Nat}{\suc{x+z}}{\suc{x}+z}]\end{array}} \]
\end{scriptsize}

Si nota adesso che basta prendere
\[ w_0 \equiv \mathrm{id}(\suc{x})\] poiché $x+0 = x$ e quindi $\suc{x+0} = \suc{x}$, e inoltre $\suc{x}+0 = \suc{x}$. Poi, dato che $\suc{\suc{x+z}} = \suc{x+\suc{z}}$ e $\suc{\suc{x}+z} = \suc{x}+\suc{z}$, applicando opportunamente conversioni e sostituzioni, è sufficiente porre
\[ w_1(z, r) \equiv \mathrm{El}_{\mathrm{Id}}(r, (x'').\mathrm{id}(\suc{x''}))\]
Per cui
\[w \equiv \mathrm{El}_{\mathrm{Nat}}(a, \mathrm{id}(\suc{x}), (z,r).\mathrm{El}_{\mathrm{Id}}(r, (x'').\mathrm{id}(\suc{x''})) \]

\subsubsection{Esercizio completo in Coq}
L'esercizio può essere risolto usando entrambe le uguaglianze.
\paragraph{Leibniz}
\lstinputlisting[language=Coq]{res/code/uguaglianzaProposizionale/es8Leibniz.v}
\paragraph{Gentzen}
\lstinputlisting[language=Coq]{res/code/uguaglianzaProposizionale/es8Gentzen.v}

\subsection{Esercizio 9}
\begin{thm}
	La regola di eliminazione del tipo uguaglianza proposizionale alla Leibniz rappresenta davvero una definizione per induzione sul tipo dei suoi elementi?
\end{thm}
No, infatti la regola di eliminazione non corrisponde direttamente alla regola di introduzione. Si nota, infatti, che la regola di introduzione crea un elemento canonico $id(a)~\in~Id(A,a,a)$, mentre l'eliminazione avviene su un elemento $x~\in~A$ dato nel contesto (e quindi, intuitivamente, passato come ``parametro''). Altri tipi di uguaglianza proposizionale, come quello a là Martin-Lof, cercano di mitigare questa disparità.

\subsection{Esercizio 10}
\begin{thm}
	Definire l'insieme vuoto $N_0$.
\end{thm}
\lstinputlisting[language=Coq]{res/code/uguaglianzaProposizionale/es10.v}

\subsection{Esercizio 11}
\begin{thm}
	Nella teoria dei tipi con tipo singoletto, uguaglianza proposizionale alla Leibniz, liste, booleano e somma disgiunta, quale tipo è definibile a partire dagli altri?
\end{thm}
Il tipo booleano è definibile a partire dalla somma disgiunta e dal singoletto, infatti:
\lstinputlisting[language=Coq]{res/code/uguaglianzaProposizionale/es11.v}
Dove $T \equiv inl (star)$ e $F \equiv inr (star)$.
