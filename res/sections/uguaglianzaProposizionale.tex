\newpage
\section{Tipo dell'uguaglianza proposizionale alla Gentzen/Lawvere e alla Leibniz}
\subsection{Esercizio 3}
\begin{thm}[L'uguaglianza di Leibniz rispetta la sostituibilità degli uguali]\label{thm:ug3} Trovare un proof-term
	\[ \mathbf{pf}\in P(b)\ [a\in A, b\in A, z\in P(a), w\in \Id{A}{a}{b}]\]
\end{thm}
\proof
Basta applicare l'eliminatore dell'uguaglianza al tipo $C(x,y) \equiv P(x)\to P(y)$, infatti
\begin{scriptsize}
	\[ \infer{p \equiv \mathrm{El}_{\mathrm{Id}}(w, (x).\lambda y. y)) \in P(a)\to P(b)\ [a \in A, b\in A, z\in P(a), w\in \Id{A}{a}{b}]}{\begin{array}{l}P(x)\to P(y) type\ [\Gamma, x \in A, y \in A] \\ \lambda y. y \in P(x) \to P(x)  [\Gamma, x \in A] \\
		a \in A [\Gamma], \quad b \in A [\Gamma] \\
		w \in \Id{A}{a}{b}\ [\Gamma] \end{array}} \]
\end{scriptsize}
per cui 
\[ \mathbf{pf} \equiv \mathrm{Ap}(p, z) \in P(b) \]
\endproof

Quindi il proof-term \[ \mathbf{pf}\in P(b)\ [a\in A, b\in A, z\in P(a), w\in \Id{A}{a}{b}]\] può essere derivato con l'uguaglianza proposizionale alla Leibniz.

\subsection{Esercizio 4}
\begin{thm}[L'uguaglianza di Leibniz è transitiva]
	Trovare un proof-term
	\[ \mathbf{pf} \in \Id{A}{a}{c}\ [a\in A, b\in A, c\in A, w_1 \in \Id{A}{a}{b}, w_2 \in \Id{A}{b}{c} ] \]
\end{thm}

\proof
Segue da \ref{thm:ug3} in quanto si trova
\[ \mathbf{pf} \in P(c)\ [b\in A, c \in A, w_1 \in P(b), w_2 \in \Id{A}{b}{c}] \]
ponendo
\[ P(c) \equiv \Id{A}{a}{c} \qquad P(b) \equiv \Id{A}{a}{b} \]
e aggiungendo $a\in A$ al contesto.
\endproof

Quindi il proof-term \[ \mathbf{pf} \in \Id{A}{a}{c}\ [a\in A, b\in A, c\in A, w_1 \in \Id{A}{a}{b}, w_2 \in \Id{A}{b}{c} ] \] può essere derivato con l'uguaglianza proposizionale alla Leibniz.

\subsection{Esercizio 8}
\subsubsection{Parte 1 - Commutatività della somma}
\begin{thm}[La somma di due naturali è commutativa]
	Trovare un proof-term del tipo
	\[ \mathrm{comm}(a, b) \in \Id{Nat}{a+b}{b+a}\ [a\in Nat,\ b\in Nat] \]
\end{thm}
\proof
Definiamo il proof-term per induzione sull'addendo $b$.\\
Posto $\Gamma \equiv a\in Nat,\ b\in Nat$

\begin{scriptsize}
	\[ \infer{\mathrm{comm}(a, b) \equiv \mathrm{El}_{Nat}(b, d, (x, y).e(x,y)) \in \Id{Nat}{a+b}{b+a}\ [\Gamma]}{\begin{array}{ll} b\in Nat\ [\Gamma] & \Id{Nat}{a+x}{x+a}\ type\ [\Gamma,\ x\in Nat] \\ d\in \Id{Nat}{a+0}{0+a}\ [\Gamma] & e(x, y)\in \Id{Nat}{a+\suc{x}}{\suc{x}+a}\ [\Gamma, x\in Nat, y\in \Id{Nat}{a+x}{x+a}]\end{array}} \]
\end{scriptsize}

Occorre quindi definire 
\[ d\in \Id{Nat}{a+0}{0+a}\ [\Gamma] \]
Notiamo intanto che

\[ \infer[conv)]{d\in \Id{Nat}{a+0}{0+a}\ [\Gamma]}
{	d\in \Id{Nat}{a}{0+a}\ [\Gamma] & 
	\infer[sym)]{\Id{Nat}{a}{0+a} = \Id{Nat}{a+0}{0+a}\ type\ [\Gamma]}
	{
		\infer[sub-eq-typ)]{\Id{Nat}{a+0}{0+a}=\Id{Nat}{a}{0+a}\ type\ [\Gamma]}
		{a+0 = a\in Nat\ [\Gamma] & \Id{Nat}{z}{0+a}\ type\ [\Gamma, z\in Nat]}
	}
}
\]
per cui è sufficiente dimostrare che esiste
\[ d\in \Id{Nat}{a}{0+a}\ [\Gamma]\]
Procediamo per induzione su $a\in Nat$.

\begin{scriptsize}
	\[ \infer{d \equiv \mathrm{El}_{Nat}(a, d_1, (x, y).e_1(x,y)) \in \Id{Nat}{a}{0+a}\ [\Gamma]}{\begin{array}{ll} a\in Nat\ [\Gamma] & \Id{Nat}{x}{0+x}\ type\ [\Gamma,\ x\in Nat] \\ d_1\in \Id{Nat}{0}{0+0}\ [\Gamma] & e_1(x, y)\in \Id{Nat}{\suc{x}}{0+\suc{x}}\ [\Gamma, x\in Nat, y\in \Id{Nat}{x}{0+x}]\end{array}} \]
\end{scriptsize}
dove $d_1 \equiv \mathrm{id}(0)$, infatti
\[ \infer[conv)]{\mathrm{id}(0)\in \Id{Nat}{0}{0+0}\ [\Gamma]}
{	\mathrm{id}(0)\in \Id{Nat}{0}{0}\ [\Gamma] & 
	\infer[sym)]{\Id{Nat}{0}{0} = \Id{Nat}{0}{0+0}\ type\ [\Gamma]}
	{
		\infer[sub-eq-typ)]{\Id{Nat}{0}{0+0}=\Id{Nat}{0}{0}\ type\ [\Gamma]}
		{0+0 = 0\in Nat\ [\Gamma] & \Id{Nat}{0}{z}\ type\ [\Gamma, z\in Nat]}
	}
}
\]
mentre sappiamo che
\[ \mathrm{El}_{Id}(y, (x'').\mathrm{id}(\suc{x''})) \in \Id{Nat}{\suc{x}}{\suc{0+x}}\]
e per conversione, dato che $0+\suc{x} = \suc{0+x}$, prendiamo
\[ e_1(x, y) \equiv \mathrm{El}_{Id}(y, (x'').\mathrm{id}(\suc{x''})) \]
pertanto
\[d \equiv \mathrm{El}_{Nat}(a, \mathrm{id}(0), (x, y).\mathrm{El}_{Id}(y, (x'').\mathrm{id}(\suc{x''}))) \in \Id{Nat}{a}{0+a}\ [\Gamma]\]

Mostriamo adesso l'esistenza del proof-term 

\[e(x, y)\in \Id{Nat}{a+\suc{x}}{\suc{x}+a}\ [\Gamma, x\in Nat, y\in \Id{Nat}{a+x}{x+a}]\]

Dato che $a+\suc{x} = \suc{a+x}$, applicando $sub-eq-typ),\ sym),\ conv)$ ci basta trovare un proof-term

\[e(x, y)\in \Id{Nat}{\suc{a+x}}{\suc{x}+a}\ [\Gamma, x\in Nat, y\in \Id{Nat}{a+x}{x+a}]\]

L'ipotesi $y \in \Id{Nat}{a+x}{x+a}$ ci consente di avere il proof-term
\[ \mathrm{El}_{\mathrm{Id}}(y, (x'').\mathrm{id}(\suc{x''})) \in \Id{Nat}{\suc{a+x}}{\suc{x+a}}\]

Per cui, se dimostrassimo che esiste $w \in \Id{Nat}{\suc{x+a}}{\suc{x}+a}$ potremmo concludere per la transitività dell'uguaglianza di Leibniz. Tale proof-term si può ottenere per induzione su $a\in Nat$.

Posto $\Delta = \Gamma, x\in Nat, y\in \Id{Nat}{a+x}{x+a}$

\begin{scriptsize}
	\[ \infer{w \equiv \mathrm{El}_{Nat}(a, w_0, (x, y).w_1(x,y)) \in \Id{Nat}{\suc{x+a}}{\suc{x}+a}\ [\Delta]}{\begin{array}{l}
		\Id{Nat}{\suc{z+x}}{\suc{x}+z}\ type\ [\Delta] \\ a\in Nat\ [\Delta] \\  w_0\in \Id{Nat}{\suc{x+0}}{\suc{x}+0}\ [\Delta] \\ w_1(z, r)\in \Id{Nat}{\suc{x+\suc{z}}}{\suc{x}+\suc{z}}\ [\Delta z\in Nat, r\in \Id{Nat}{\suc{x+z}}{\suc{x}+z}]\end{array}} \]
\end{scriptsize}

Si nota adesso che basta prendere
\[ w_0 \equiv \mathrm{id}(\suc{x})\] poiché $x+0 = x$ e quindi $\suc{x+0} = \suc{x}$, e inoltre $\suc{x}+0 = \suc{x}$. Poi, dato che $\suc{\suc{x+z}} = \suc{x+\suc{z}}$ e $\suc{\suc{x}+z} = \suc{x}+\suc{z}$, applicando opportunamente conversioni e sostituzioni, è sufficiente porre
\[ w_1(z, r) \equiv \mathrm{El}_{\mathrm{Id}}(r, (x'').\mathrm{id}(\suc{x''}))\]
Per cui
\[w \equiv \mathrm{El}_{\mathrm{Nat}}(a, \mathrm{id}(\suc{x}), (z,r).\mathrm{El}_{\mathrm{Id}}(r, (x'').\mathrm{id}(\suc{x''})) \]

\endproof