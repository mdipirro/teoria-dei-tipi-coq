\newpage
\section{Come tradurre la logica predicativa con uguaglianza in teoria dei tipi}
\subsection{Esercizio 1}
\begin{thm}[Assioma di Scelta] Si mostri che la traduzione CHM della logica nella teoria dei tipi rende valido l'assioma di scelta
	\[ (\forall x\in A)\ (\exists y\in B)\  R(x, y)\ \to\ (\exists f\in A\to B)\ (\forall x\in A)\ R(x, f(x))\ \mathrm{true}\ [\ ] \]
	
\end{thm}

\proof
Occorre trovare un proof-term $\textbf{pf}$ del tipo
\[\textbf{pf} \equiv \lambda w.p(w) \in \Pi_{x\in A}\Sigma_{y\in B}R(x,y)\to\ \Sigma_{f\in {\Pi_A B}}\Pi_{x\in A}R(x, f(x))\ [\ ] \]
per cui basta avere un proof-term del tipo (introduzione di $\to$)
\[ p(w) \in \Sigma_{f\in {\Pi_A B}}\Pi_{x\in A}R(x, f(x))\ [w\in \Pi_{x\in A}\Sigma_{y\in B}R(x,y) ] \]
Avendo $w\in \Pi_{x\in A}\Sigma_{y\in B}R(x,y)$, usando l'eliminatore del $\Pi$ si ottiene 
\[ \mathrm{Ap}(w, x)\in \Sigma_{y\in B}R(x,y) \]
e quindi, usando l'eliminatore del $\Sigma$,
\[ \pi_1(\mathrm{Ap}(w,x))\in B \qquad \pi_2(\mathrm{Ap}(w,x))\in R(x, \pi_1(\mathrm{Ap}(w, x)))\]
pertanto possiamo definire
\[ p(w) \equiv \langle \lambda x. \pi_1(\mathrm{Ap}(w, x)), \lambda x.\pi_2(\mathrm{Ap}(w, x))\rangle \]
\endproof

\subsection{Esercizio 2}
\begin{thm}
	Dire se il seguente giudizio è derivabile
	\begin{scriptsize}
		\[(\forall x \in A)(\exists y \in B)R(x, y) \to (\forall x_1\in A)(\forall x_2\in A)(\exists y_1\in B)(\exists y_2\in B)(( R(x_1, y_1)\ \wedge\ R(x_2, y_2))\ \wedge\  (x_1 =_A x_2\ \to y_1 =_B y_2))\]
	\end{scriptsize}
\end{thm}
\proof
Dobbiamo trovare un proof-term del seguente tipo
\begin{scriptsize}
	\[ \Pi_{x\in A}\Sigma_{y\in B}R(x, y) \to \Pi_{x_1\in A}\Pi_{x_2\in A}\Sigma_{y_1\in B}\Sigma_{y_2\in B}(R(x_1, y_1)\times R(x_2, y_2)\times (\Id{A}{x_1}{x_2} \to \Id{B}{y_1}{y_2}))\]
\end{scriptsize}
Avendo un elemento di $\Pi_{x\in A}\Sigma_{y\in B}R(x,y)$, per via della validità dell'Assioma di Scelta possiamo ottenere un elemento $f \in A \to B$ tale per cui $R(x, f(x))$ risulta abitato per ogni $x \in A$. A questo punto, dati $x_1, x_2 \in A$ possiamo scegliere
\[ y_1 := f(x_1) \qquad y_2 := f(x_2) \]
in modo che $R(x_1, y_1)$ e $R(x_2, y_2)$ siano abitati, e inoltre, dato un proof-term
\[ w \in \Id{A}{x_1}{x_2}\]
otteniamo
\[ \mathrm{El}_{\mathrm{Id}}(w, (x').\mathrm{id}(f(x'))) \in \Id{B}{f(x_1)}{f(x_2)}\] ovvero un proof-term di 
\[ \Id{B}{y_1}{y_2} \]
come richiesto.
\endproof

\subsection{Esercizio 3}
\begin{thm}
	Supposto di rappresentare il tipo dei booleani come $Boole\equiv N_1~+~N_1$, con $0\equiv inl(*)$ e $1\equiv inr(*)$, e supposto che esista un proof-term $dis~\in~N_0~[z~\in~Id(Boole, 0, 1)]$, si dimostri che,dati due tipi \textit{A type} e \textit{B type} esiste una funzione
	\[f(z)~\in~\Sigma_{x\in {Boole}}~((Id(Boole,x,1)\rightarrow A)~\times~(Id(Boole,x,0)\rightarrow B))~[z~\in~A~+~B]\] \\
	ed anche una funzione
	\[g(w)~\in~A~+~B~[w~\in~\Sigma_{x\in {Boole}}~((Id(Boole,x,1)\rightarrow A)~\times~(Id(Boole,x,0)\rightarrow B))]\]
\end{thm}
\lstinputlisting[language=Coq]{res/code/traduzioneLogica/es3.v}

\subsection{Esercizio 4}
\begin{thm}
	Le funzioni $f(z)$ e $g(w)$ stabiliscono un isomorfismo nel senso della definizione 9.6 tra $A~+~B$ e $\Sigma_{x\in {Boole}}~((Id(Boole,x,1)\rightarrow A)~\times~(Id(Boole,x,0)\rightarrow B))$?
\end{thm}
Si, infatti è possibile definire due proof-terms \textbf{pf1} e \textbf{pf2} tali che $pf_1~\in~Id(A~+~B, x,g(f(x)))~[\Gamma,~x~\in~A~+~B]$ e $pf_2~\in~Id(\Sigma_{x\in {Boole}}~((Id(Boole,x,1)\rightarrow A)~\times~(Id(Boole,x,0)\rightarrow B)), y,f(g(y)))~[\Gamma,~y~\in~\Sigma_{x\in {Boole}}~((Id(Boole,x,1)\rightarrow A)~\times~(Id(Boole,x,0)\rightarrow B))]$
\lstinputlisting[language=Coq]{res/code/traduzioneLogica/es4.v}

\subsection{Esercizio 6}
\begin{thm}
	Si noti che in teoria dei tipi vale il seguente principio chiamato \textit{existence property under a context}: se $\exists y~\phi(y)~true~[\Gamma]$ allora esiste un proof-term \textbf{pf} tale che $\phi(pf)~true~[\Gamma]$.
\end{thm}
\lstinputlisting[language=Coq]{res/code/traduzioneLogica/es6.v}