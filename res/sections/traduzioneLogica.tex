\newpage
\section{Come tradurre la logica predicativa con uguaglianza in teoria dei tipi}
\subsection{Esercizio 1}
\begin{thm}[Assioma di Scelta] Si mostri che la traduzione CHM della logica nella teoria dei tipi rende valido l'assioma di scelta
	\[ (\forall x\in A)\ (\exists y\in B)\  R(x, y)\ \to\ (\exists f\in A\to B)\ (\forall x\in A)\ R(x, f(x))\ \mathrm{true}\ [\ ] \]
	
\end{thm}

\proof
Occorre trovare un proof-term $\textbf{pf}$ del tipo
\[\textbf{pf} \equiv \lambda w.p(w) \in \Pi_{x\in A}\Sigma_{y\in B}R(x,y)\to\ \Sigma_{f\in {\Pi_A B}}\Pi_{x\in A}R(x, f(x))\ [\ ] \]
per cui basta avere un proof-term del tipo (introduzione di $\to$)
\[ p(w) \in \Sigma_{f\in {\Pi_A B}}\Pi_{x\in A}R(x, f(x))\ [w\in \Pi_{x\in A}\Sigma_{y\in B}R(x,y) ] \]
Avendo $w\in \Pi_{x\in A}\Sigma_{y\in B}R(x,y)$, usando l'eliminatore del $\Pi$ si ottiene 
\[ \mathrm{Ap}(w, x)\in \Sigma_{y\in B}R(x,y) \]
e quindi, usando l'eliminatore del $\Sigma$,
\[ \pi_1(\mathrm{Ap}(w,x))\in B \qquad \pi_2(\mathrm{Ap}(w,x))\in R(x, \pi_1(\mathrm{Ap}(w, x)))\]
pertanto possiamo definire
\[ p(w) \equiv \langle \lambda x. \pi_1(\mathrm{Ap}(w, x)), \lambda x.\pi_2(\mathrm{Ap}(w, x))\rangle \]
\endproof

\subsection{Esercizio 2}
\begin{thm}
	Dire se il seguente giudizio è derivabile
	\begin{scriptsize}
		\[(\forall x \in A)(\exists y \in B)R(x, y) \to (\forall x_1\in A)(\forall x_2\in A)(\exists y_1\in B)(\exists y_2\in B)(( R(x_1, y_1)\ \wedge\ R(x_2, y_2))\ \wedge\  (x_1 =_A x_2\ \to y_1 =_B y_2))\]
	\end{scriptsize}
\end{thm}
\proof
Dobbiamo trovare un proof-term del seguente tipo
\begin{scriptsize}
	\[ \Pi_{x\in A}\Sigma_{y\in B}R(x, y) \to \Pi_{x_1\in A}\Pi_{x_2\in A}\Sigma_{y_1\in B}\Sigma_{y_2\in B}(R(x_1, y_1)\times R(x_2, y_2)\times (\Id{A}{x_1}{x_2} \to \Id{B}{y_1}{y_2}))\]
\end{scriptsize}
Avendo un elemento di $\Pi_{x\in A}\Sigma_{y\in B}R(x,y)$, per via della validità dell'Assioma di Scelta possiamo ottenere un elemento $f \in A \to B$ tale per cui $R(x, f(x))$ risulta abitato per ogni $x \in A$. A questo punto, dati $x_1, x_2 \in A$ possiamo scegliere
\[ y_1 := f(x_1) \qquad y_2 := f(x_2) \]
in modo che $R(x_1, y_1)$ e $R(x_2, y_2)$ siano abitati, e inoltre, dato un proof-term
\[ w \in \Id{A}{x_1}{x_2}\]
otteniamo
\[ \mathrm{El}_{\mathrm{Id}}(w, (x').\mathrm{id}(f(x'))) \in \Id{B}{f(x_1)}{f(x_2)}\] ovvero un proof-term di 
\[ \Id{B}{y_1}{y_2} \]
come richiesto.
\endproof

\subsection{Esercizio 3}
\begin{thm}
	Supposto di rappresentare il tipo dei booleani come $Boole\equiv N_1~+~N_1$, con $0\equiv inl(*)$ e $1\equiv inr(*)$, e supposto che esista un proof-term $dis~\in~N_0~[z~\in~Id(Boole, 0, 1)]$, si dimostri che,dati due tipi \textit{A type} e \textit{B type} esiste una funzione
	\[f(z)~\in~\Sigma_{x\in {Boole}}~((Id(Boole,x,1)\rightarrow A)~\times~(Id(Boole,x,0)\rightarrow B))~[z~\in~A~+~B]\] \\
	ed anche una funzione
	\[g(w)~\in~A~+~B~[w~\in~\Sigma_{x\in {Boole}}~((Id(Boole,x,1)\rightarrow A)~\times~(Id(Boole,x,0)\rightarrow B))]\]
\end{thm}
\lstinputlisting[language=Coq]{res/code/traduzioneLogica/es3.v}

\subsection{Esercizio 4}
\begin{thm}
	Le funzioni $f(z)$ e $g(w)$ stabiliscono un isomorfismo nel senso della definizione 9.6 tra $A~+~B$ e $\Sigma_{x\in {Boole}}~((Id(Boole,x,1)\rightarrow A)~\times~(Id(Boole,x,0)\rightarrow B))$?
\end{thm}
Si, infatti è possibile definire due proof-terms \textbf{pf1} e \textbf{pf2} tali che \[pf_1~\in~Id(A~+~B, x,g(f(x)))~[\Gamma,~x~\in~A~+~B]\] e 
\begin{scriptsize}
\[pf_2~\in~Id(\Sigma_{x\in {Boole}}~((Id(Boole,x,1)\rightarrow A)~\times~(Id(Boole,x,0)\rightarrow B)), y,f(g(y)))\]
\[[\Gamma,~y~\in~\Sigma_{x\in {Boole}}~((Id(Boole,x,1)\rightarrow A)~\times~(Id(Boole,x,0)\rightarrow B))]\]
\end{scriptsize}
\lstinputlisting[language=Coq]{res/code/traduzioneLogica/es4.v}
Tuttavia, in Coq, è impossibile provare l'esistenza di $pf2$ a causa della nozione troppo debole di uguaglianza estensionale delle funzioni. Quest'ultima è definita come
\begin{lstlisting}[language=Coq]
	forall f g : A -> B, (forall x : A, f x = g x) -> f = g.
\end{lstlisting}
ma non è disponibile in Coq per default. 

\subsection{Esercizio 5}
\begin{thm}
	Quali regole di inferenza del calcolo dei sequenti della logica classica predicativa nelle dispense in http://www.math.unipd.it/~maietti/2lo16.html sono valide in teoria dei tipi, nel senso che conservano la validità dei sequenti coinvolti nella regola?
\end{thm}
\lstinputlisting[language=Coq]{res/code/traduzioneLogica/es5.v}

\subsection{Esercizio 6}
\begin{thm}
	Si noti che in teoria dei tipi vale il seguente principio chiamato \textit{existence property under a context}: se $\exists y~\phi(y)~true~[\Gamma]$ allora esiste un proof-term \textbf{pf} tale che $\phi(pf)~true~[\Gamma]$.
\end{thm}
\lstinputlisting[language=Coq]{res/code/traduzioneLogica/es6.v}

\subsection{Esercizio 7}
\begin{thm}
	Per ogni connettivo e quantificatore della logica predicativa con uguaglianza si descrivano le relative regole di formazione, introduzione, eliminazione ed uguaglianza che risultano ammissibili in teoria dei tipi.
\end{thm}
Siano $\Phi$ e $\Psi$ due proposizioni.
\paragraph{And} \mbox{} \\
\AxiomC{$\Phi~prop~[\Gamma]$}
\AxiomC{$\Psi~prop~[\Gamma]$}
\RightLabel{$F-\wedge$}
\BinaryInfC{$\Phi~\wedge~\Psi~prop~[\Gamma]$}
\DisplayProof\qquad
\AxiomC{$a~\in~\Phi$}
\AxiomC{$b~\in~\Psi$}
\RightLabel{$I-\wedge$}
\BinaryInfC{$<a, b>~\in~\Phi~\wedge~\Psi~[\Gamma]$}
\DisplayProof

\vspace{0.2in}
\AxiomC{$p~\in~\Phi~\wedge~\Psi~[\Gamma]$}
\RightLabel{$E_1-\wedge$}
\UnaryInfC{$fst(p)~\in~B~[\Gamma]$}
\DisplayProof\qquad
\AxiomC{$p~\in~\Phi~\wedge~\Psi~[\Gamma]$}
\RightLabel{$E_2-\wedge$}
\UnaryInfC{$snd(p)~\in~C~[\Gamma]$}
\DisplayProof

\vspace{0.2in}
\AxiomC{$a~=~a'~\in~\Phi~[\Gamma]$}
\AxiomC{$b~=~b'~\in~\Psi~[\Gamma]$}
\RightLabel{$U_1-\wedge$}
\BinaryInfC{$<a,b>~=~<a',b'>~\in~\Phi~\wedge~\Psi~[\Gamma]$}
\DisplayProof

\vspace{0.2in}
\AxiomC{$p~=~q~\in~\Phi~\wedge~\Psi~[\Gamma]$}
\RightLabel{$U_2-\wedge$}
\UnaryInfC{$fst(p)~=~fst(q)~\in~\Phi~\wedge~\Psi~[\Gamma]$}
\DisplayProof\qquad
\AxiomC{$p~=~q~\in~\Phi~\wedge~\Psi~[\Gamma]$}
\RightLabel{$U_3-\wedge$}
\UnaryInfC{$snd(p)~=~snd(q)~\in~\Phi~\wedge~\Psi~[\Gamma]$}
\DisplayProof

\vspace{0.2in}
\AxiomC{$a~\in~\Phi~[\Gamma]$}
\AxiomC{$b~\in~\Psi~[\Gamma]$}
\RightLabel{$C_1-\wedge$}
\BinaryInfC{$fst(<a,b>)~=~a~\in~\Phi~[\Gamma]$}
\DisplayProof\qquad
\AxiomC{$a~\in~\Phi~[\Gamma]$}
\AxiomC{$b~\in~\Psi~[\Gamma]$}
\RightLabel{$C_2-\wedge$}
\BinaryInfC{$snd(<a,b>)~=~b~\in~\Psi~[\Gamma]$}
\DisplayProof

\paragraph{Or} \mbox{} \\
\AxiomC{$\Phi~prop~[\Gamma]$}
\AxiomC{$\Psi~prop~[\Gamma]$}
\RightLabel{$F-\vee$}
\BinaryInfC{$\Phi~\vee~\Psi~prop~[\Gamma]$}
\DisplayProof\qquad
\AxiomC{$a~\in~\Phi~[\Gamma]$}
\RightLabel{$I_1-\vee$}
\UnaryInfC{$inl(a)~\in~\Phi~\vee~\Psi~[\Gamma]$}
\DisplayProof\qquad
\AxiomC{$a~\in~\Psi~[\Gamma]$}
\RightLabel{$I_2-\vee$}
\UnaryInfC{$inr(a)~\in~\Phi~\vee~\Psi~[\Gamma]$}
\DisplayProof

\vspace{0.2in}
\AxiomC{$p~\in~\Phi~\vee~\Psi~[\Gamma]$}
\AxiomC{$X~prop~[\Gamma]$}
\AxiomC{$d(u)~\in~X~[\Gamma,~u~\in~\Phi]$}
\AxiomC{$c(w)~\in~X~[\Gamma,~w~\in~\Psi]$}
\RightLabel{$E-\vee$}
\QuaternaryInfC{$case(p,~(u).d,~(w).c)~\in~X~[\Gamma]$}
\DisplayProof

\vspace{0.2in}
\AxiomC{$a~=~a'~\in~\Phi~[\Gamma]$}
\RightLabel{$U_1-\vee$}
\UnaryInfC{$inl(a)~=~inl(a')~\in~\Phi~\vee~\Psi~[\Gamma]$}
\DisplayProof\qquad
\AxiomC{$b~=~b'~\in~\Psi~[\Gamma]$}
\RightLabel{$U_2-\vee$}
\UnaryInfC{$inr(b)~=~inr(b')~\in~\Phi~\vee~\Psi~[\Gamma]$}
\DisplayProof

\vspace{0.2in}
\AxiomC{$X~prop~[\Gamma]$}
\AxiomC{$p~=~q~\in~\Phi~\vee~\Psi~[\Gamma]$}
\AxiomC{$c(w)~=~d(u)~\in~X~[\Gamma,~w~\in~\Psi,~u~\in~\Phi]$}
\RightLabel{$U_3-\vee$}
\TrinaryInfC{$case(p,~(u).d,~(w).c)~=~case(q,~(u).d,~(w).c)~\in~\Phi~\vee~\Psi~[\Gamma]$}
\DisplayProof

\vspace{0.2in}
\AxiomC{$a~\in~\Phi~[\Gamma]$}
\AxiomC{$X~prop~[\Gamma]$}
\AxiomC{$d(u)~\in~X~[\Gamma,~u~\in~\Phi]$}
\AxiomC{$c(w)~\in~X~[\Gamma,~w~\in~\Psi]$}
\RightLabel{$C_1-\vee$}
\QuaternaryInfC{$case(inl(a),~(u).d,~(w).c)~=~d(u)~\in~\Phi~[\Gamma]$}
\DisplayProof

\vspace{0.2in}
\AxiomC{$b~\in~\Psi~[\Gamma]$}
\AxiomC{$X~prop~[\Gamma]$}
\AxiomC{$d(u)~\in~X~[\Gamma,~u~\in~\Phi]$}
\AxiomC{$c(w)~\in~X~[\Gamma,~w~\in~\Psi]$}
\RightLabel{$C_2-\vee$}
\QuaternaryInfC{$case(inr(b),~(u).d,~(w).c)~=~c(w)~\in~\Psi~[\Gamma]$}
\DisplayProof

\paragraph{Implicazione} \mbox{} \\
\AxiomC{$\Phi~prop~[\Gamma]$}
\AxiomC{$\Psi~prop~[\Gamma]$}
\RightLabel{$F-\to$}
\BinaryInfC{$\Phi~\to~\Psi~prop~[\Gamma]$}
\DisplayProof\qquad
\AxiomC{$b(u)~\in~\Psi~[\Gamma,~u~\in~\Phi]$}
\RightLabel{$I-\to$}
\UnaryInfC{$\lambda u.b(u)~\in~\Phi~\to~\Psi~[\Gamma]$}
\DisplayProof

\vspace{0.2in}
\AxiomC{$f~\in~\Phi~\to~\Psi~[\Gamma]$}
\AxiomC{$p~\in~\Phi~[\Gamma]$}
\RightLabel{$E-\to$}
\BinaryInfC{$Ap(f, p)~\in~\Psi~[\Gamma]$}
\DisplayProof\qquad
\AxiomC{$b(u)~=~b'(u)~\in~\Psi~[\Gamma,~u~\in~\Phi]$}
\RightLabel{$U_1-\to$}
\UnaryInfC{$\lambda u.b(u)~=~\lambda u.b'(u)~\in~\Phi~\to~\Psi~[\Gamma]$}
\DisplayProof

\vspace{0.2in}
\AxiomC{$f~=~f'~\in~\Phi~\to~\Psi~[\Gamma]$}
\AxiomC{$p~=~p'~\in~\Phi~[\Gamma]$}
\RightLabel{$U_2-\to$}
\BinaryInfC{$Ap(f, p)~=~Ap(f',p')~\in~\Psi~[\Gamma]$}
\DisplayProof

\vspace{0.2in}
\AxiomC{$a~\in~\Phi~[\Gamma]$}
\AxiomC{$b(u)~\in~\Psi~[\Gamma,~u~\in~\Phi]$}
\RightLabel{$C-\to$}
\BinaryInfC{$Ap(\lambda u.b(u),~a)~=~b(a)~\in~\Psi~[\Gamma]$}
\DisplayProof

\paragraph{Per ogni} \mbox{} \\
\AxiomC{$\Phi(x)~prop~[\Gamma,~x~\in~A]$}
\RightLabel{$F-\forall$}
\UnaryInfC{$(\forall x~\in~A)\Phi(x)~prop~[\Gamma]$}
\DisplayProof\qquad
\AxiomC{$A~type~[\Gamma]$}
\AxiomC{$b(w)~\in~\Phi(w)~[\Gamma,~w~\in~A]$}
\RightLabel{$I-\forall$}
\BinaryInfC{$\lambda w.b(w)~\in~(\forall x~\in~A)\Phi(x)~[\Gamma]$}
\DisplayProof

\vspace{0.2in}
\AxiomC{$A~type~[\Gamma]$}
\AxiomC{$f~\in~(\forall x~\in~A)\Phi(x)~[\Gamma]$}
\AxiomC{$t~\in~A~[\Gamma]$}
\RightLabel{$E-\forall$}
\TrinaryInfC{$Ap(f, t)~\in~\Phi(t)~[\Gamma]$}
\DisplayProof

\vspace{0.2in}
\AxiomC{$A~type~[\Gamma]$}
\AxiomC{$b(w)~=b'(w)~\in~\Phi(w)~[\Gamma,~w~\in~A]$}
\RightLabel{$U_1-\forall$}
\BinaryInfC{$\lambda w.b(w)~=~\lambda w.b'(w)~\in~(\forall x~\in~A)\Phi(x)~[\Gamma]$}
\DisplayProof

\vspace{0.2in}
\AxiomC{$A~type~[\Gamma]$}
\AxiomC{$f~=~f'~\in~(\forall x~\in~A)\Phi(x)~[\Gamma]$}
\AxiomC{$t~=~t'~\in~A~[\Gamma]$}
\RightLabel{$U_2-\forall$}
\TrinaryInfC{$Ap(f, t)~=~Ap(f',t')~\in~\Phi(t)~[\Gamma]$}
\DisplayProof

\vspace{0.2in}
\AxiomC{$A~type~[\Gamma]$}
\AxiomC{$t~\in~A~[\Gamma]$}
\AxiomC{$b(w)~\in~\Phi(w)~[\Gamma,~w~\in~A]$}
\RightLabel{$C-\forall$}
\TrinaryInfC{$Ap(\lambda w.b(w),~a)~=~b(a)~\in~\Phi(a)~[\Gamma]$}
\DisplayProof

\paragraph{Esiste} \mbox{} \\
\AxiomC{$\Phi(x)~prop~[\Gamma,~x~\in~A]$}
\RightLabel{$F-\exists$}
\UnaryInfC{$(\exists x~\in~A)\Phi(x)~prop~[\Gamma]$}
\DisplayProof\qquad
\AxiomC{$A~type~[\Gamma]$}
\AxiomC{$t~\in~A~[\Gamma]$}
\AxiomC{$p~\in~\Phi(t)~[\Gamma]$}
\RightLabel{$I-\exists$}
\TrinaryInfC{$<t, p>~\in~(\exists x~\in~A)\Phi(x)~[\Gamma]$}
\DisplayProof

\vspace{0.2in}
\AxiomC{$A~type~[\Gamma]$}
\AxiomC{$\Psi~prop~[\Gamma]$}
\AxiomC{$p~\in~(\exists x~\in~A)\Phi(x)~[\Gamma]$}
\AxiomC{$q(w,~y)~\in~\Psi~[\Gamma,~w~\in~A,~y~\in~\Phi(x)]$}
\RightLabel{$E-\exists$}
\QuaternaryInfC{$unpack(p, (w,~y)q)~\in~\Psi~[\Gamma]$}
\DisplayProof

\vspace{0.2in}
\AxiomC{$A~type~[\Gamma]$}
\AxiomC{$t~=~t'~\in~A~[\Gamma]$}
\AxiomC{$p~=~p'~\in~\Phi(t)~[\Gamma]$}
\RightLabel{$U_1-\exists$}
\TrinaryInfC{$unpack(<t, p>, )~=~<t',p'>~\in~(\exists x~\in~A)\Phi(x)~[\Gamma]$}
\DisplayProof

\begin{scriptsize}
	\vspace{0.2in}
	\AxiomC{$A~type~[\Gamma]$}
	\AxiomC{$\Psi~prop~[\Gamma]$}
	\AxiomC{$p~=~p'~\in~(\exists x~\in~A)\Phi(x)~[\Gamma]$}
	\AxiomC{$q(w,~y)~=~q'(w,~y)~\in~\Psi~[\Gamma,~w~\in~A,~y~\in~\Phi(x)]$}
	\RightLabel{$U_2-\exists$}
	\QuaternaryInfC{$unpack(p, (w,~y)q)~=~unpack(p', (w,~y)q')~\in~\Psi~[\Gamma]$}
	\DisplayProof
	
	\vspace{0.2in}
	\AxiomC{$A~type~[\Gamma]$}
	\AxiomC{$t~\in~A~[\Gamma]$}
	\AxiomC{$p~\in~\Phi(t)~[\Gamma]$}
	\AxiomC{$q(w,~y)~\in~\Psi~[\Gamma,~w~\in~A,~y~\in~\Phi(x)]$}
	\RightLabel{$C-\exists$}
	\QuaternaryInfC{$unpack(<t, p>, q)~=~q(t,p)~\in~\Psi~[\Gamma]$}
	\DisplayProof
\end{scriptsize}

\paragraph{Uguaglianza} \mbox{} \\
\AxiomC{$\Phi~prop~[\Gamma]$}
\AxiomC{$a~\in~\Psi~[\Gamma]$}
\AxiomC{$b~\in~\Psi~[\Gamma]$}
\RightLabel{$F-Id$}
\TrinaryInfC{$\Id(\Phi, a, b)~prop~[\Gamma]$}
\DisplayProof\qquad
\AxiomC{$a~\in~Phi$}
\RightLabel{$I-Id$}
\UnaryInfC{$id(a)~\in~Id(\Phi,a,a)~[\Gamma]$}
\DisplayProof
	
\begin{scriptsize}
	\vspace{0.2in}
	\AxiomC{$C(x,~y)~type~[\Gamma,~x~\in~\Phi,~y~\in~\Phi]$}
	\AxiomC{$a~\in~\Phi~[\Gamma]$}
	\AxiomC{$b~\in~\Phi~[\Gamma]$}
	\AxiomC{$p~\in~Id(\Phi,a,b)~[\Gamma]$}
	\AxiomC{$c(x)~\in~C(x,x)~[\Gamma,~x\in~\Phi]$}
	\RightLabel{$E-Id$}
	\QuinaryInfC{$El(p,~(x).c)~\in~C(a,b)~[\Gamma]$}
	\DisplayProof
\end{scriptsize}

\vspace{0.2in}
\AxiomC{$a~=~b~\in~\Phi~[\Gamma,~a\in~\Phi,~b\in~\Phi]$}
\RightLabel{$U_1-Id$}
\UnaryInfC{$id(a)~=~id(b)~\in~Id(\Phi,a,b)~[\Gamma,~a\in~\Phi,~b\in~\Phi]$}
\DisplayProof

\begin{scriptsize}
	\vspace{0.2in}
	\AxiomC{$C(x,~y)~type~[\Gamma,~x~\in~\Phi,~y~\in~\Phi]$}
	\AxiomC{$a~=~b~\in~\Phi~[\Gamma]$}
	\AxiomC{$p~=~q~\in~\Phi~=~\Phi~[\Gamma]$}
	\AxiomC{$c(x)~=~d(x)~\in~C(x,x)~[\Gamma,~x\in~\Phi,~c~\in~C(x,x),~d~\in~C(x,x)]$}
	\RightLabel{$U_2-Id$}
	\QuaternaryInfC{$El(p,~(a).c)=~sse(q,~(b).d)~\in~C(a,~p)~[\Gamma]$}
	\DisplayProof
	
	\vspace{0.2in}
	\AxiomC{$C(x,~y)~type~[\Gamma,~x~\in~\Phi,~y~\in~\Phi]$}
	\AxiomC{$a~\in~\Phi~[\Gamma]$}
	\AxiomC{$c(x)~\in~C(x,x)~[\Gamma,~x\in~\Phi]$}
	\RightLabel{$C-Id$}
	\TrinaryInfC{$El(id(a),~(x).c)~=c(a)~\in~C(a,a)~[\Gamma]$}
	\DisplayProof
\end{scriptsize}

\subsection{Esercizio 8}
\begin{thm}
	Per ogni connettivo e quantificatore della logica predicativa con uguaglianza si descrivano le relative regole con giudizi del tipo $\phi~prop~[\Gamma]$ e $\phi~true~[\Gamma]$ di formazione, introduzione ed eliminazione che risultano ammissibili in teoria dei tipi.
\end{thm}
\paragraph{And} \mbox{} \\
\AxiomC{$\Psi~prop$}
\AxiomC{$\Phi~prop$}
\RightLabel{$F-\wedge$}
\BinaryInfC{$\Psi \wedge \Phi~prop$}
\DisplayProof\qquad
\AxiomC{$\Psi~true$}
\AxiomC{$\Phi~true$}
\RightLabel{$I-\wedge$}
\BinaryInfC{$\Psi \wedge \Phi~true$}
\DisplayProof\qquad
\AxiomC{$\Psi \wedge \Phi~true$}
\RightLabel{$E_L-\wedge$}
\UnaryInfC{$\Psi~true$}
\DisplayProof

\vspace{0.2in}
\AxiomC{$\Psi \wedge \Phi~true$}
\RightLabel{$E_R-\wedge$}
\UnaryInfC{$\Phi~true$}
\DisplayProof

\paragraph{Or} \mbox{} \\
\AxiomC{$\Psi~prop$}
\AxiomC{$\Phi~prop$}
\RightLabel{$F-\vee$}
\BinaryInfC{$\Psi \vee \Phi~prop$}
\DisplayProof\qquad
\AxiomC{$\Psi~true$}
\RightLabel{$I_L-\vee$}
\UnaryInfC{$\Psi \vee \Phi~true$}
\DisplayProof\qquad
\AxiomC{$\Phi~true$}
\RightLabel{$I_R-\vee$}
\UnaryInfC{$\Psi \vee \Phi~true$}
\DisplayProof

\vspace{0.2in}
\AxiomC{$\Psi \vee \Phi~true$}
\AxiomC{}
\RightLabel{$u$}
\UnaryInfC{$\Psi~true$}
\noLine
\UnaryInfC{$\vdots$}
\noLine
\UnaryInfC{$\chi~true$}
\AxiomC{}
\RightLabel{$w$}
\UnaryInfC{$\Phi~true$}
\noLine
\UnaryInfC{$\vdots$}
\noLine
\UnaryInfC{$\chi~true$}
\RightLabel{$\vee E^{u,w}$}
\TrinaryInfC{$\chi~true$}
\DisplayProof

\paragraph{Implicazione} \mbox{} \\
\AxiomC{$\Psi~prop$}
\AxiomC{$\Phi~prop$}
\RightLabel{$F-\to$}
\BinaryInfC{$\Psi \to \Phi~prop$}
\DisplayProof\qquad
\AxiomC{}
\RightLabel{$u$}
\UnaryInfC{$\Psi~true$}
\noLine
\UnaryInfC{$\vdots$}
\noLine
\UnaryInfC{$\Phi~true$}
\RightLabel{$I^u-\to$}
\UnaryInfC{$\Psi\to\Phi~true$}
\DisplayProof\qquad
\AxiomC{$\Psi\to\Phi~true$}
\AxiomC{$\Psi~true$}
\RightLabel{$E-\to$}
\BinaryInfC{$\Phi~true$}
\DisplayProof

\paragraph{Negazione} \mbox{} \\
\AxiomC{$\Phi~prop$}
\RightLabel{$F-\neg$}
\UnaryInfC{$\neg\Phi~prop$}
\DisplayProof\qquad
\AxiomC{$\Psi\to\perp~true$}
\RightLabel{$I-\neg$}
\UnaryInfC{$\neg\Psi~true$}
\DisplayProof\qquad
\AxiomC{$\neg\Psi~true$}
\RightLabel{$E-\neg$}
\UnaryInfC{$\Psi\to\perp~true$}
\DisplayProof

\paragraph{Per ogni} \mbox{} \\
\AxiomC{$\Psi~prop$}
\RightLabel{$F-\forall$}
\UnaryInfC{$\forall~x.\Psi~prop$}
\DisplayProof\qquad
\AxiomC{[a/x]$\Psi~true$}
\RightLabel{$I^{a}-\forall$}
\UnaryInfC{$\forall x. \Psi(x)~true$}
\DisplayProof\qquad
\AxiomC{$\forall x. \Psi(x)~true$}
\RightLabel{$E-\forall$}
\UnaryInfC{[t/x]$\Psi(x)~true$}
\DisplayProof

\paragraph{Esiste} \mbox{} \\
\AxiomC{$\Psi~prop$}
\RightLabel{$F-\exists$}
\UnaryInfC{$\exists x.\Psi(x)~prop$}
\DisplayProof\qquad
\AxiomC{[t/x]$\Psi(t)~true$}
\RightLabel{$I-\exists$}
\UnaryInfC{$\exists x. \Psi(x)~true$}
\DisplayProof\qquad
\AxiomC{$\exists x. \Psi(x)~true$}
\AxiomC{}
\RightLabel{$u$}
\UnaryInfC{[a/x]$\Psi(x)~true$}
\noLine
\UnaryInfC{$\vdots$}
\noLine
\UnaryInfC{$\chi~true$}
\RightLabel{$E^{a,u}-\exists$}
\BinaryInfC{$\chi~true$}
\DisplayProof

\paragraph{Uguaglianza} \mbox{} \\
\AxiomC{$\Psi~prop$}
\RightLabel{$F-=$}
\UnaryInfC{$\Psi~=~\Psi~prop$}
\DisplayProof\qquad
\AxiomC{$\Phi\to\Psi~true$}
\AxiomC{$\Psi\to\Phi~true$}
\RightLabel{$I-=$}
\BinaryInfC{$\Psi=\Phi~true$}
\DisplayProof\qquad

\vspace{0.2in}
\AxiomC{$\Psi=\Phi~true$}
\RightLabel{$E_1-=$}
\UnaryInfC{$\Psi\to\Phi~true$}
\DisplayProof\qquad
\AxiomC{$\Psi=\Phi~true$}
\RightLabel{$E_2-=$}
\UnaryInfC{$\Phi\to\Psi~true$}
\DisplayProof

\subsection{Esercizio 9}
\begin{thm}
	Verificare che tutti gli assiomi di Peano sono validi in teoria dei tipi con le regole finora introdotte e interpretando le formule logiche come descritto.
\end{thm}
\lstinputlisting[language=Coq]{res/code/traduzioneLogica/es9.v}

\subsection{Lemma 9.2}
\begin{lem}
	L'interpretazione CHM di un sequente di formule $\Sigma~\vdash~\Delta$ con variabili nel contesto tipato $\Gamma$ è \textit{valida in teoria dei tipi} se e solo se esiste un \textbf{proof-term pf} tale che 
	\[pf~\in~(\Sigma^{\&})^I~\to~(\Delta^{\vee})^I~[\Gamma]\]
	
	è derivabile.
\end{lem}

\proof \mbox{} \\

$\to$ L'interpretazione CHM di un sequente $\Sigma~\vdash~\Delta$ è il tipo $(\Delta^{\vee})^I~type~[\Gamma,~(\Sigma^{\&})^I]$, e il sequente è valido se e solo se esiste un proof-term $pf'$ tale che 
$pf'~\in~(\Delta^{\vee})^I~[\Gamma,~(\Sigma^{\&})^I]$. Visto che l'interpretazione CHM è valida per ipotesi, si ha che $(\Delta^{\vee})^I$ è abitato sulla base di $\Gamma$ e di $(\Sigma^{\&})^I$, ed esiste un tale $pf'$.

Questo rende vera la premessa di $I-\to$, cioè $c(x)~\in~(\Delta^{\vee})^I~[\Gamma,~x\in(\Sigma^{\&})^I]$, ovvero $c(x)~\equiv~pf'$. Di conseguenza il tipo $(\Sigma^{\&})^I~\to~(\Delta^{\vee})^I$ è abitato, perché è valida la premessa della sua regola di introduzione. È quindi possibile definire $pf$ a partire da quest'ultima, cioè $pf~\equiv~\lambda x^{(\Sigma^{\&})^I}.c(x)$.

\vspace{0.2in}
$\leftarrow$ Si ha per ipotesi un proof-term $pf$ tale che $pf~\in~(\Sigma^{\&})^I~\to~(\Delta^{\vee})^I~[\Gamma]$. $pf$ è stato derivato utilizzando $I-\to$, e quindi è del tipo $\lambda x^{(\Sigma^{\&})^I}.c(x)$, dato un $c(x)~\in~C~[\Gamma,~x~\in~(\Sigma^{\&})^I]$. Di conseguenza l'interpretazione è valida, perché il proof-term $pf'$ abitante del tipo 
$(\Delta^{\vee})^I~type~[\Gamma,~(\Sigma^{\&})^I]$ è esattamente $c(x)$.

\endproof
