\newpage
\section{Somma Disgiunta}
\subsection{Esercizio 1}
\begin{thm}
	Definire il tipo booleano B come tipo semplice e provare che è rappresentabile da $N_1 + N_1$.
\end{thm}
\proof
Le regole sono le seguenti:

	\begin{scriptsize}
		\vspace{0.2in}
		\AxiomC{$\Gamma~cont$}
		\RightLabel{$I_1-B$}
		\UnaryInfC{$0 \in B~[\Gamma]$}
		\DisplayProof\qquad
		\AxiomC{$\Gamma~cont$}
		\RightLabel{$I_2-B$}
		\UnaryInfC{$1 \in B~[\Gamma]$}
		\DisplayProof\qquad
		\AxiomC{$\Gamma~cont$}
		\RightLabel{$F-B$}
		\UnaryInfC{$B~type~[\Gamma]$}
		\DisplayProof
		
		\vspace{0.2in}
		\AxiomC{$M(x)~type~[\Gamma,~x~\in~B]$}
		\AxiomC{$z~\in~B~[\Gamma]$}
		\AxiomC{$a~\in~M(0)~[\Gamma]$}
		\AxiomC{$b~\in~M(1)~[\Gamma]$}
		\RightLabel{$E-B$}
		\QuaternaryInfC{$\mathrm{El}_B(z, a, b) \in M(z)~[\Gamma]$}
		\DisplayProof
		
		\vspace{0.2in}
		\AxiomC{$M(z)~type~[\Gamma,~z~\in~B]$}
		\AxiomC{$a~\in~M(0)~[\Gamma]$}
		\AxiomC{$b~\in~M(1)~[\Gamma]$}
		\RightLabel{$C_1-B$}
		\TrinaryInfC{$\mathrm{El}_B(0, a, b)~=~a \in M(0)~[\Gamma]$}
		\DisplayProof
		
		\vspace{0.2in}
		\AxiomC{$M(z)~type~[\Gamma,~z~\in~B]$}
		\AxiomC{$a~\in~M(0)~[\Gamma]$}
		\AxiomC{$b~\in~M(1)~[\Gamma]$}
		\RightLabel{$C_2-B$}
		\TrinaryInfC{$\mathrm{El}_B(1, a, b)~=~b \in M(1)~[\Gamma]$}
		\DisplayProof
	\end{scriptsize}

\vspace{0.5in}Riscritte per utilizzare $N_1~+~N_1$:

\begin{scriptsize}
	\vspace{0.2in}
	\AxiomC{$a \in N_1~[\Gamma]$}
	\RightLabel{$I_1-B$}
	\UnaryInfC{$inl(a) \in B~[\Gamma]$}
	\DisplayProof\qquad
	\AxiomC{$b \in N_1~[\Gamma]$}
	\RightLabel{$I_2-B$}
	\UnaryInfC{$inr(b) \in B~[\Gamma]$}
	\DisplayProof\qquad
	\AxiomC{$N_1~type~[\Gamma]$}
	\RightLabel{$F-B$}
	\UnaryInfC{$B~type~[\Gamma]$}
	\DisplayProof
	
	\vspace{0.2in}
	\AxiomC{$M(z)~type~[\Gamma,~z~\in~B]$}
	\AxiomC{$w~\in~B~[\Gamma]$}
	\AxiomC{$a(x)~\in~M(inl(x))~[\Gamma,x~\in~N_1]$}
	\AxiomC{$b(x)~\in~M(inr(x))~[\Gamma,x~\in~N_1]$}
	\RightLabel{$E-B$}
	\QuaternaryInfC{$\mathrm{El}_B(w, a, b) \in M(w)~[\Gamma]$}
	\DisplayProof
	
	\vspace{0.2in}
	\AxiomC{$M(z)~type~[\Gamma,~z~\in~B]$}
	\AxiomC{$c~\in~N_1~[\Gamma]$}
	\AxiomC{$a(x)~\in~M(inl(x))~[\Gamma,x~\in~N_1]$}
	\AxiomC{$b(x)~\in~M(inr(x))~[\Gamma,x~\in~N_1]$}
	\RightLabel{$C_1-B$}
	\QuaternaryInfC{$\mathrm{El}_B(inl(c), a, b)~=~a \in M(inl(c))~[\Gamma]$}
	\DisplayProof
	
	\vspace{0.2in}
	\AxiomC{$M(z)~type~[\Gamma,~z~\in~B]$}
	\AxiomC{$c~\in~N_1~[\Gamma]$}
	\AxiomC{$a(x)~\in~M(inl(x))~[\Gamma,x~\in~N_1]$}
	\AxiomC{$b(x)~\in~M(inr(x))~[\Gamma,x~\in~N_1]$}
	\RightLabel{$C_2-B$}
	\QuaternaryInfC{$\mathrm{El}_B(inr(c), a, b)~=~b \in M(inr(c))~[\Gamma]$}
	\DisplayProof
\end{scriptsize}

\vspace{0.5in}In Coq:
\lstinputlisting[language=Coq]{res/code/sommaDisgiunta/es1.v}
\endproof

\subsection{Esercizio 2}
\begin{thm}
	Definire la funzione predecessore tale che $p(0) = 0$ e $p(succ(n)) = n$.
\end{thm}
\lstinputlisting[language=Coq]{res/code/sommaDisgiunta/es2.v}

\subsection{Esercizio 3}
\begin{thm}
	Dimostrare che esistono i termini $dec(z)~\in Nat~[z~\in~N_1~+~Nat]$ e $cod(w)~\in~N_1~+~Nat~[w~\in~Nat]$ tali che, per ogni numerale $n$ vale $dec(cod(0))~=~0$ e $dec(cod(n))~=~n$.
\end{thm}
\lstinputlisting[language=Coq]{res/code/sommaDisgiunta/es3.v}