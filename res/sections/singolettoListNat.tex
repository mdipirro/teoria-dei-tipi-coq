\newpage
\setcounter{section}{2}
\section{Tipi singoletto, delle liste e dei numeri naturali}
\subsection{Esercizio importante}
\textit{Si formulino le regole dell'uguaglianza per il tipo dei numeri naturali.}
\begin{center}
	\AxiomC{$m~=~m'~\in~Nat~[\Gamma]$}
	\RightLabel{$I_2-EQ-Nat$}
	\UnaryInfC{$succ(m)~=~succ(m')~\in~Nat~[\Gamma]$}
	\DisplayProof
	
	\vspace{0.2in}
	\begin{small}
		\AxiomC{$m=m'\in Nat[\Gamma]$}
		\AxiomC{$e(x,y)=e'(x,y)\in D(succ(x))[\Gamma,x\in Nat, y\in D(x)]$}
		\noLine
		\UnaryInfC{$D(x)type[\Gamma,x\in Nat]$}
		\AxiomC{$d=d'\in D(0)[\Gamma]$}
		\RightLabel{$E-EQ-Nat$}
		\TrinaryInfC{$El_{Nat}(m,d,e)~=~El_{Nat}(m',d',e')~\in~D(m)~[\Gamma]$}
		\DisplayProof
	\end{small}
\end{center}

\subsection{Proposizione 3.1}
\begin{prop}
	La teoria dei tipi che include tutte le regole finora descritte eccetto quelle di sostituzione e di indebolimento rende ammissibili le regole di sostituzione della sezione 3.0.1 e quelle di indebolimento della sezione 2.3.3.
\end{prop}
\proof \mbox{} \\
\textbf{Singoletto}

\textbf{ind-type-1} Siano $\pi_1$ e $\pi_2$ i seguenti alberi di derivazione, rispettivamente:

\begin{center}
	\AxiomC{$\Gamma~cont$}
	\RightLabel{S}
	\UnaryInfC{$N_1~type~[\Gamma]$}
	\DisplayProof\qquad
	\AxiomC{$\Gamma~cont$}
	\RightLabel{S}
	\UnaryInfC{$N_1~type~[\Gamma]$}
	\RightLabel{Fc}
	\UnaryInfC{$\Gamma,~z\in N_1~cont$}
	\DisplayProof
\end{center}

Se si applica S a $\pi_2$ si ottiene $N_1~type~[\Gamma,z\in N_1]$, cioè la conclusione di $ind-type-1$:

\begin{center}
	\AxiomC{$\pi_1$}
	\noLine
	\UnaryInfC{$N_1~type~[\Gamma]$}
	\AxiomC{$\pi_2$}
	\noLine
	\UnaryInfC{$\Gamma,~z\in N_1~cont$}
	\BinaryInfC{$N_1~type~[\Gamma,z\in N_1]$}
	\DisplayProof
\end{center}

\textbf{ind-type-2} Siano $\pi_1$ e $\pi_2$ i seguenti alberi di derivazione, rispettivamente:

\begin{center}
	\AxiomC{$\Gamma~cont$}
	\RightLabel{S}
	\UnaryInfC{$N_1~type~[\Gamma]$}
	\RightLabel{Ref}
	\UnaryInfC{$N_1=N_1 type [\Gamma]$}
	\DisplayProof\qquad
	\AxiomC{$\Gamma~cont$}
	\RightLabel{S}
	\UnaryInfC{$N_1~type~[\Gamma]$}
	\RightLabel{Fc}
	\UnaryInfC{$\Gamma,~z\in N_1~cont$}
	\DisplayProof
\end{center}

Se si applicano S e Ref a $\pi_2$ si ottiene $N_1=N_1~type~[\Gamma,z\in N_1]$, cioè la conclusione di $ind-type-2$:

\begin{center}
	\AxiomC{$\pi_1$}
	\noLine
	\UnaryInfC{$N_1=N_1 type [\Gamma]$}
	\AxiomC{$\pi_2$}
	\noLine
	\UnaryInfC{$\Gamma,~z\in N_1~cont$}
	\BinaryInfC{$N_1=N_1~type~[\Gamma,z\in N_1]$}
	\DisplayProof
\end{center}

\textbf{ind-type-3} Siano $\pi_1$ e $\pi_2$ i seguenti alberi di derivazione, rispettivamente:

\begin{center}
	\AxiomC{$\Gamma~cont$}
	\RightLabel{S}
	\UnaryInfC{$N_1~type~[\Gamma]$}
	\RightLabel{I-S}
	\UnaryInfC{$\star\in N_1[\Gamma]$}
	\DisplayProof\qquad
	\AxiomC{$\Gamma~cont$}
	\RightLabel{S}
	\UnaryInfC{$N_1~type~[\Gamma]$}
	\RightLabel{Fc}
	\UnaryInfC{$\Gamma,~z\in N_1~cont$}
	\DisplayProof
\end{center}

Se si applica I-S a $\pi_2$ si ottiene $\star\in N_1 [\Gamma,z\in N_1]$, cioè la conclusione di $ind-type-3$:

\begin{center}
	\AxiomC{$\pi_1$}
	\noLine
	\UnaryInfC{$\star\in N_1[\Gamma]$}
	\AxiomC{$\pi_2$}
	\noLine
	\UnaryInfC{$\Gamma,~z\in N_1~cont$}
	\BinaryInfC{$\star\in N_1 [\Gamma,z\in N_1]$}
	\DisplayProof
\end{center}

\textbf{ind-type-4} Siano $\pi_1$ e $\pi_2$ i seguenti alberi di derivazione, rispettivamente:

\begin{center}
	\AxiomC{$\Gamma~cont$}
	\RightLabel{S}
	\UnaryInfC{$N_1~type~[\Gamma]$}
	\RightLabel{I-S}
	\UnaryInfC{$\star\in N_1[\Gamma]$}
	\RightLabel{Ref}
	\UnaryInfC{$\star=\star\in N_1[\Gamma]$}
	\DisplayProof\qquad
	\AxiomC{$\Gamma~cont$}
	\RightLabel{S}
	\UnaryInfC{$N_1~type~[\Gamma]$}
	\RightLabel{Fc}
	\UnaryInfC{$\Gamma,~z\in N_1~cont$}
	\DisplayProof
\end{center}

Se si applicano S, I-S e Ref a $\pi_2$ si ottiene $\star=\star\in N_1 [\Gamma,z\in N_1]$, cioè la conclusione di $ind-type-4$:

\begin{center}
	\AxiomC{$\pi_1$}
	\noLine
	\UnaryInfC{$\star=\star\in N_1[\Gamma]$}
	\AxiomC{$\pi_2$}
	\noLine
	\UnaryInfC{$\Gamma,~z\in N_1~cont$}
	\BinaryInfC{$\star=\star\in N_1 [\Gamma,z\in N_1]$}
	\DisplayProof
\end{center}

\textbf{sub-typ} Dato che il tipo singoletto non è dipendente, i tipi $A_1,\dots,A_n(a_1,\dots,a_{n-1}), C$ sono tutti $N_1$. Siano $\pi_1$ e $\pi_2$ i seguenti alberi di derivazione, rispettivamente:

\begin{center}
	\AxiomC{$\Gamma~cont$}
	\RightLabel{S}
	\UnaryInfC{$N_1~type~[\Gamma]$}
	\RightLabel{S}
	\UnaryInfC{$\star\in N_1[\Gamma]$}
	\DisplayProof\qquad
	\AxiomC{$\Gamma~cont$}
	\RightLabel{S}
	\UnaryInfC{$N_1~type~[\Gamma]$}
	\RightLabel{Fc}
	\UnaryInfC{$\Gamma,~x_1\in N_1~cont$}
	\RightLabel{S}
	\UnaryInfC{$N_1~type~[\Gamma,~x_1\in N_1]$}
	\UnaryInfC{\vdots}
	\UnaryInfC{$N_1~type~[\Gamma,x_1\in N_1,\dots,x_n\in N_1]$}
	\DisplayProof
\end{center}

Applicando $ind-typ-1$ a $\pi_2$ si ottiene $N_1~type~[\Gamma]$, cioè la conclusione di $sub-typ$:

\begin{center}
	\AxiomC{$\pi_1$}
	\noLine
	\UnaryInfC{$\star\in N_1[\Gamma]$}
	\AxiomC{$\pi_2$}
	\noLine
	\UnaryInfC{$N_1~type~[\Gamma,x_1\in N_1,\dots,x_n\in N_1]$}
	\BinaryInfC{$N_1~type~[\Gamma]$}
	\DisplayProof
\end{center}

\textbf{sub-eq-typ} Dato che il tipo singoletto non è dipendente, i tipi $A_1,\dots,A_n(a_1,\dots,a_{n-1}), C$ sono tutti $N_1$. Siano $\pi_1$ e $\pi_2$ i seguenti alberi di derivazione, rispettivamente:

\begin{center}
	\AxiomC{$\Gamma~cont$}
	\RightLabel{S}
	\UnaryInfC{$N_1~type~[\Gamma]$}
	\RightLabel{I-S}
	\UnaryInfC{$\star\in N_1[\Gamma]$}
	\RightLabel{Ref}
	\UnaryInfC{$\star=\star\in N_1[\Gamma]$}
	\DisplayProof\qquad
	\AxiomC{$\Gamma~cont$}
	\RightLabel{S}
	\UnaryInfC{$N_1~type~[\Gamma]$}
	\RightLabel{Fc}
	\UnaryInfC{$\Gamma,~x_1\in N_1~cont$}
	\RightLabel{S}
	\UnaryInfC{$N_1~type~[\Gamma,~x_1\in N_1]$}
	\UnaryInfC{\vdots}
	\UnaryInfC{$N_1~type~[\Gamma,x_1\in N_1,\dots,x_n\in N_1]$}
	\DisplayProof
\end{center}

Applicando Ref a $\pi_2$ si ottiene $N_1=N_1~type~[\Gamma]$, cioè la conclusione di $sub-eq-typ$:

\begin{center}
	\AxiomC{$\pi_1$}
	\noLine
	\UnaryInfC{$\star=\star\in N_1[\Gamma]$}
	\AxiomC{$\pi_2$}
	\noLine
	\UnaryInfC{$N_1~type~[\Gamma,x_1\in N_1,\dots,x_n\in N_1]$}
	\BinaryInfC{$N_1=N_1~[\Gamma]$}
	\DisplayProof
\end{center}

%TODO altre regole per singoletto

\textbf{List e Nat}

La dimostrazione di ammissibilità per i tipi $List$ e $Nat$ è uguale a quella per il tipo $N_1$.
\endproof

\subsection{Esercizio 1}
\begin{thm}
	Dati i tipi singoletto e delle liste è possibile definire il tipo dei numeri naturali Nat?
\end{thm}
Si è possibile, è sufficiente considerare un numero naturale come la concatenazione di tanti elementi singoletto. 

\subsection{Esercizio 2}
\begin{thm}
	Definire l'operazione di \textit{append} tra due liste di tipo A, tale che $append(x, nil) = x$.
\end{thm}
\lstinputlisting[language=Coq]{res/code/singolettoListNat/es2.v}

\subsection{Esercizio 3}
\begin{thm}
	Definire le operazioni \textit{back}, \textit{front}, \textit{last}, \textit{first} per le liste.
\end{thm}
\lstinputlisting[language=Coq]{res/code/singolettoListNat/es3.v}

\subsection{Esercizio 4}
\begin{thm}
	Definire un'operazione binaria \textit{bin} sui naturali.
\end{thm}
\lstinputlisting[language=Coq]{res/code/singolettoListNat/es4.v}

\subsection{Esercizio 5}
\begin{thm}
	Definire l'operazione di addizione tale che $x + 0 = x$
\end{thm}
\lstinputlisting[language=Coq]{res/code/singolettoListNat/es5.v}

\subsection{Esercizio 6}
\begin{thm}
	Definire l'operazione di addizione tale che $0 + y = y$
\end{thm}
\lstinputlisting[language=Coq]{res/code/singolettoListNat/es6.v}

\subsection{Esercizio 7}
\begin{thm}
	Definire l'operazione di moltiplicazione tale che $x * 0 = 0$
\end{thm}
\lstinputlisting[language=Coq]{res/code/singolettoListNat/es7.v}

\subsection{Esercizio 8}
\begin{thm}
	Definire l'operazione di reverse sulle liste di tipo A.
\end{thm}
\lstinputlisting[language=Coq]{res/code/singolettoListNat/es8.v}