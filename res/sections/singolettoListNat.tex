\section{Tipi singoletto, delle liste e dei numeri naturali}
\subsection{Esercizio 1}
\begin{thm}
	Dati i tipi singoletto e delle liste è possibile definire il tipo dei numeri naturali Nat?
\end{thm}
Si è possibile, è sufficiente considerare un numero naturale come la concatenazione di tanti elementi singoletto. 

\subsection{Esercizio 2}
\begin{thm}
	Definire l'operazione di \textit{append} tra due liste di tipo A, tale che $append(x, nil) = x$.
\end{thm}
\lstinputlisting[language=Coq]{res/code/singolettoListNat/es2.v}

\subsection{Esercizio 3}
\begin{thm}
	Definire le operazioni \textit{back}, \textit{front}, \textit{last}, \textit{first} per le liste.
\end{thm}
\lstinputlisting[language=Coq]{res/code/singolettoListNat/es3.v}

\subsection{Esercizio 4}
\begin{thm}
	Definire un'operazione binaria \textit{bin} sui naturali.
\end{thm}
\lstinputlisting[language=Coq]{res/code/singolettoListNat/es4.v}

\subsection{Esercizio 5}
\begin{thm}
	Definire l'operazione di addizione tale che $x + 0 = x$
\end{thm}
\lstinputlisting[language=Coq]{res/code/singolettoListNat/es5.v}

\subsection{Esercizio 6}
\begin{thm}
	Definire l'operazione di addizione tale che $0 + y = y$
\end{thm}
\lstinputlisting[language=Coq]{res/code/singolettoListNat/es6.v}

\subsection{Esercizio 7}
\begin{thm}
	Definire l'operazione di moltiplicazione tale che $x * 0 = 0$
\end{thm}
\lstinputlisting[language=Coq]{res/code/singolettoListNat/es7.v}

\subsection{Esercizio 8}
\begin{thm}
	Definire l'operazione di reverse sulle liste di tipo A.
\end{thm}
\lstinputlisting[language=Coq]{res/code/singolettoListNat/es8.v}