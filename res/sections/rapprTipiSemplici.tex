\section{Rappresentazione dei tipi semplici in teoria dei tipi dipendenti}
\subsection{Lemma 8.1}
\begin{lem}
	Supposto che $A~type~[]$ e $B~type~[]$ siano derivabili nella teoria dei tipi con le regole finora introdotte, allora in tal teoria è derivabile anche $\Pi_{x\in A}B~type~[]$, e lo si può usare per interpretare il tipo semplice $A\to B$ con le sue regole di introduzione, eliminazione e relativa $\beta$-conversione.
\end{lem}

\proof
%TODO
\endproof

\subsection{Lemma 8.2}
\begin{lem}
	Supposto che $A~type~[]$ e $B~type~[]$ siano derivabili nella teoria dei tipi con le regole finora introdotte, allora in tal teoria è derivabile anche $\Sigma_{x\in A}B~type~[]$, e lo si può usare per interpretare il tipo semplice $A\times B$ con le sue regole di introduzione, eliminazione e relative $\beta$-conversioni.
\end{lem}

\proof

Dato che $A$ e $B$ sono tipi semplici, la premessa di $F-\Sigma$ non necessita della variabile $x\in A~[\Gamma]$. La regola diventa:

\begin{center}
	\AxiomC{$A~type~[\Gamma]$}
	\AxiomC{$B~type~[\Gamma]$}
	\RightLabel{$F-\times$}
	\BinaryInfC{$A\times B~type~[\Gamma]$}
	\DisplayProof
\end{center}

\vspace{0.3in}
Per lo stesso motivo anche la regola di introduzione non necessita delle premesse riguardanti il tipo dipendente C, e diventa:

\begin{center}
	\AxiomC{$a\in A~[\Gamma]$}
	\AxiomC{$b\in B~[\Gamma]$}
	\RightLabel{$I-\times$}
	\BinaryInfC{$<a,b>\in A\times B~[\Gamma]$}
	\DisplayProof
\end{center}

\vspace{0.3in}
La regola di eliminazione del tipo $\Sigma$ è la seguente:

\begin{center}
	\AxiomC{$M(z)type[\Gamma,z\in\Sigma_{x\in A}C(x)]$}
	\AxiomC{$d\in\Sigma_{x\in A}C(x[\Gamma])$}
	\AxiomC{$m(x,y)\in M(<x,y>)[\Gamma,x\in B,y\in C(x)]$}
	\RightLabel{$E-\Sigma$}
	\TrinaryInfC{$El_{\Sigma}(d,m)\in M(d)[\Gamma]$}
	\DisplayProof
\end{center}

In questa regola non è più necessaria la premessa $M(z)type[\Gamma,z\in\Sigma_{x\in A}C(x)]$, dato che siamo in presenza di tipi semplici. Inoltre $d\in\Sigma_{x\in A}C(x[\Gamma])$ diventa $d\in B\times C [\Gamma]$, per lo stesso motivo. L'ultima premessa, $m(x,y)\in M(<x,y>)[\Gamma,x\in B,y\in C(x)]$, cambia a seconda del termine $m$ che consideriamo. Ci sono due casi, corrispondenti alle due proiezioni della coppia rappresentata dall'elemento canonico. Se $m(x,y)\equiv x$ si ha $M(<x,y>)\equiv B$ e $y\in C$ nel contesto. Se invece $m(x,y)\equiv y$, allora $M(<x,y>)\equiv C$. $m$ diventa quindi un termine costante e può essere definito direttamente all'interno dell'eliminatore, ottenendo due varianti della regola di eliminazione:
\AxiomC{$d\in B\times C [\Gamma]$}
\RightLabel{$E_1-\times$}
\UnaryInfC{$El(d,~(x,y).x)\in B[\Gamma]$}
\DisplayProof\qquad
\AxiomC{$d\in B\times C [\Gamma]$}
\RightLabel{$E_2-\times$}
\UnaryInfC{$El(d,~(x,y).y)\in C[\Gamma]$}
\DisplayProof

Possiamo rinominare, per chiarezza, i due eliminatori come segue: $El(d,~(x,y).x)\equiv\pi_1(d)$ e $El(d,~(x,y).y)\equiv\pi_2(d)$.

%TODO: beta conversioni
\endproof

\subsection{Esercizio 1}
\begin{thm}
	Si dimostri che se $a~=~b~\in~A~[\Gamma]$ è derivabile nella teoria dei tipi con le regole finora introdotte allora esiste un proof-term \textbf{pf} tale che $pf~\in~Id(A,a,b)~[\Gamma]$.
\end{thm}
\lstinputlisting[language=Coq]{res/code/rappreTipiSemplici/es1.v}

\subsection{Esercizio 2}
\begin{thm}
	La $\eta$-conversione del tipo prodotto, ovvero
	\[<\pi_1(z),\pi_2(z)>~=~z~\in~A~\times~B~[z~\in~A~\times~B]\]
	è derivabile?
	
	È derivabile la corrispondente uguaglianza proposizionale, ovvero esiste un proof-term \textbf{pf} tale che
	\[pf~\in~Id(A\times B,<\pi_1(z),\pi_2(z)>,z)~[z~\in~A~\times~B]\]
	è derivabile nella teoria dei tipi dipendenti?
\end{thm}
La $\eta$-conversione del prodotto non è derivabile in teoria dei tipi in quanto $z$ è una variabile, e quindi è in forma normale non canonica, mentre la coppia è in forma normale canonica. Qui il teorema di forma normale non serve, visto che $z$ non è termine chiuso. Infatti, se due termini sono convertibili (uguali definizionalmente), allora le loro forme normali (che in teoria dei tipi esistono e sono uniche) devono essere identiche (sintatticamente). Siccome i due termini della $\eta$-conversione sono già in forma normale, e chiaramente non identici, l’uguaglianza definizionale non può valere.

Al contrario si deriva abbastanza facilmente che esiste un proof-term per la relativa uguaglianza proposizionale.
\lstinputlisting[language=Coq]{res/code/rappreTipiSemplici/es2.v}