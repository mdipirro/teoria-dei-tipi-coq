\section{Rappresentazione dei tipi semplici in teoria dei tipi dipendenti}
\subsection{Esercizio 1}
\begin{thm}
	Si dimostri che se $a~=~b~\in~A~[\Gamma]$ è derivabile nella teoria dei tipi con le regole finora introdotte allora esiste un proof-term \textbf{pf} tale che $pf~\in~Id(A,a,b)~[\Gamma]$.
\end{thm}
\lstinputlisting[language=Coq]{res/code/rappreTipiSemplici/es1.v}

\subsection{Esercizio 2}
\begin{thm}
	La $\eta$-conversione del tipo prodotto, ovvero
	\[<\pi_1(z),\pi_2(z)>~=~z~\in~A~\times~B~[z~\in~A~\times~B]\]
	è derivabile?
	
	È derivabile la corrispondente uguaglianza proposizionale, ovvero esiste un proof-term \textbf{pf} tale che
	\[pf~\in~Id(A\times B,<\pi_1(z),\pi_2(z)>,z)~[z~\in~A~\times~B]\]
	è derivabile nella teoria dei tipi dipendenti?
\end{thm}
\lstinputlisting[language=Coq]{res/code/rappreTipiSemplici/es2.v}