\section{Tipi somma indiciata e prodotto dipendente}
\subsection{Esercizio 1}
\begin{thm}
	Provare a scrivere le regole del prodotto cartesiano $A \times B$ di un tipo A con un tipo B.
\end{thm}
Le regole sono le seguenti:

\vspace{0.2in}
\AxiomC{$A~type~[\Gamma]$}
\AxiomC{$B~type~[\Gamma]$}
\RightLabel{$F-\times$}
\BinaryInfC{$A~\times~B~type~[\Gamma]$}
\DisplayProof\qquad
\AxiomC{$a~\in~A~[\Gamma]$}
\AxiomC{$b~\in~B~[\Gamma]$}
\RightLabel{$I-\times$}
\BinaryInfC{$<a,b>~\in~A~\times~B~[\Gamma]$}
\DisplayProof

\vspace{0.2in}
\AxiomC{$d~\in~A~\times~B~[\Gamma]$}
\RightLabel{$E_1-\times$}
\UnaryInfC{$\pi_1(d)~\in~A~[\Gamma]$}
\DisplayProof\qquad
\AxiomC{$d~\in~A~\times~B~[\Gamma]$}
\RightLabel{$E_2-\times$}
\UnaryInfC{$\pi_2(d)~\in~B~[\Gamma]$}
\DisplayProof

\vspace{0.2in}
\AxiomC{$a~\in~A~[\Gamma]$}
\AxiomC{$b~\in~B~[\Gamma]$}
\RightLabel{$\beta_1-\times$}
\BinaryInfC{$\pi_1(<a,b>)~=~a~\in~A[\Gamma]$}
\DisplayProof\qquad
\AxiomC{$a~\in~A~[\Gamma]$}
\AxiomC{$b~\in~B~[\Gamma]$}
\RightLabel{$\beta_2-\times$}
\BinaryInfC{$\pi_2(<a,b>)~=~b~\in~B[\Gamma]$}
\DisplayProof

\subsection{Esercizio 2}
\begin{thm}
	Provare a scrivere le regole del tipo delle funzioni $A \rightarrow B$ da un tipo A ad un tipo B.
\end{thm}
Le regole sono le seguenti:

\vspace{0.2in}
\AxiomC{$A~type~[\Gamma]$}
\AxiomC{$B~type~[\Gamma]$}
\RightLabel{$F-\rightarrow$}
\BinaryInfC{$A~\rightarrow~B~type~[\Gamma]$}
\DisplayProof\qquad
\AxiomC{$b(x)~\in~B~[\Gamma,~x~\in~A]$}
\RightLabel{$I-\rightarrow$}
\UnaryInfC{$\lambda~x^A.b(x)~\in~A~\rightarrow~B~[\Gamma]$}
\DisplayProof

\vspace{0.2in}
\AxiomC{$a~\in~A~[\Gamma]$}
\AxiomC{$f~\in~A~\rightarrow~B~[\Gamma]$}
\RightLabel{$E-\rightarrow$}
\BinaryInfC{$Ap(f,a)~\in~B(a)~[\Gamma]$}
\DisplayProof\qquad
\AxiomC{$a~\in~A~[\Gamma]$}
\AxiomC{$b(x)~\in~B~[\Gamma,~x~\in~A]$}
\RightLabel{$\beta C-\times$}
\BinaryInfC{$Ap(\lambda~x^A.b(x),a)~=~b(a)~\in~B~[\Gamma]$}
\DisplayProof


\subsection{Esercizio 3}
\begin{thm}
	Come rappresentare con i tipi finora descritti il tipo dei naturali positivi? E delle liste non vuote?
\end{thm}
\lstinputlisting[language=Coq]{res/code/sommaIndiciata/es3.v}

\subsection{Esercizio 4}
\begin{thm}
	Come rappresentare con i tipi finora descritti il tipo delle funzioni tra naturali positivi?
\end{thm}
\lstinputlisting[language=Coq]{res/code/sommaIndiciata/es4.v}