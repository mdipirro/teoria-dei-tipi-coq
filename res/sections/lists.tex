\newpage
\section{Lists}

Tipo list a cui si riferiscono tutti gli esercizi:

\lstinputlisting{listType.v}

\subsection{Esercizio 1 pg. 11 - appendStart}

Dato un tipo A semplice, ovvero non dipendente, per esempio A = N1, definire
l'operazione appendStart: \\

$appendStart(x, y) \in List(A)\quad[x \in List(A), y \in A]$ \\

tale che l’elemento y venga posto all'inizio della lista x.

Si traduce in:

\lstinputlisting{appendStart.v}

\subsection{Esercizio 2 pg. 12 - appendEnd}

Dato un tipo A semplice, ovvero non dipendente, per esempio A = N1, definire
l'operazione appendEnd: \\

$appendEnd(x, y) = cons(x, y) \in List(A)\quad[x \in List(A), y' \in List(A)]$ \\

tale che l’elemento y venga posto alla fine della lista x.
Si definisce la funzione per ricorsione su x: \\

$appendEnd(nil, y) = y :: nil$

$appendEnd( s :: c, y) = s :: (appendEnd c y)$ \\

e si traduce in:

\lstinputlisting{appendEnd.v}

Per rendere appendStart ed appendEnd indipendenti dal tipo della lista:

\lstinputlisting{appendGeneric.v}

Definire l'operazione append di accostamento di una lista ad un'altra di un tipo
A type [] usando le regole del tipo delle liste: \\

$append(x,y) \in List(A)\quad[x \in List(A), y \in List(A)]$ \\

Si traduce in Coq:

\lstinputlisting{append.v}

\subsection{Esercizio 3 pg. 12 - back, front, first e last}

I seguenti esercizi richiedono di scrivere funzioni che accettano tutte una
lista non vuota (diversa da nil). Viene quindi preliminarmente definito un tipo
notNil: \\

$notNil (A : Type) := \{ lt : list A | ~ lt = nil \}$ \\

Specificare il tipo e definire le operazioni:

(a) back, che di una lista non vuota ne estrae la lista meno il primo elemento

\lstinputlisting{back.v}

Nota: back accetta argomenti di tipo notNil.

(b) l'operazione front che data una lista non vuota ne estrae la lista meno
l'ultimo elemento

\lstinputlisting{front.v}

da completare

(c) l'operazione last che data una lista non vuota ne estrae l'ultimo elemento

\lstinputlisting{last.v}

da completare

(d) l'operazione first che data una lista non vuota ne estrae il primo
elemento

\lstinputlisting{first.v}

Nota: first accetta argomenti di tipo notNil.
