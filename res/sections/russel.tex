\section{Regole per l’universo dei tipi piccoli à la Russell}
\subsection{Esercizio 1}
\begin{thm} 
	\`E possibile aggiungere un'infinità di universi ove ogni universo successivo contiene un codice dell'universo precedente ed è chiuso sugli stessi costruttori di tipo del primo universo $U_0$. Provare a formularne le regole.
\end{thm}
Di seguito vengono presentate le regole sia per universi à la Russell che per universi à la Tarski.

\paragraph{Tarski} \mbox{} \\
Regole ``classiche'' modificate per tenere conto di infiniti universi:

\vspace{0.2in}
\AxiomC{$\Gamma~cont$}
\RightLabel{$F-U_i$}
\UnaryInfC{$U_i~type~[\Gamma]$}
\DisplayProof\qquad
\AxiomC{$\Gamma~cont$}
\RightLabel{$I_1-U_i$}
\UnaryInfC{$\stackrel{\wedge_i}{N_0}~\in~U_i~[\Gamma]$}
\DisplayProof\qquad
\AxiomC{$\Gamma~cont$}
\RightLabel{$I_2-U_i$}
\UnaryInfC{$\stackrel{\wedge_i}{N_1}~\in~U_i~[\Gamma]$}
\DisplayProof\qquad
\AxiomC{$\Gamma~cont$}
\RightLabel{$I_3-U_i$}
\UnaryInfC{$\stackrel{\wedge_i}{Nat}~\in~U_i~[\Gamma]$}
\DisplayProof

\vspace{0.2in}
\AxiomC{$c(x)~\in~U_i~[\Gamma,x\in T(b)]$}
\AxiomC{$b~\in~U_i~[\Gamma]$}
\RightLabel{$I_4-U_i$}
\BinaryInfC{$\stackrel{\wedge_i}{\Sigma}_{x\in b}c(x)~\in~U_i~[\Gamma]$}
\DisplayProof\qquad
\AxiomC{$c(x)~\in~U_i~[\Gamma,x\in T(b)]$}
\AxiomC{$b~\in~U_i~[\Gamma]$}
\RightLabel{$I_5-U_i$}
\BinaryInfC{$\stackrel{\wedge_i}{\Pi}_{x\in b}c(x)~\in~U_i~[\Gamma]$}
\DisplayProof

\vspace{0.2in}
\AxiomC{$b~\in~U_i~[\Gamma]$}
\AxiomC{$c~\in~U_i~[\Gamma]$}
\RightLabel{$I_6-U_i$}
\BinaryInfC{$b\stackrel{\wedge_i}{+}c~\in~U_i~[\Gamma]$}
\DisplayProof\qquad
\AxiomC{$b~\in~U_i~[\Gamma]$}
\RightLabel{$I_7-U_i$}
\UnaryInfC{$\stackrel{\wedge_i}{List}(b)~\in~U_i~[\Gamma]$}
\DisplayProof

\vspace{0.2in}
\AxiomC{$b~\in~U_i~[\Gamma]$}
\AxiomC{$c~\in~T(b)~[\Gamma]$}
\AxiomC{$d~\in~T(b)~[\Gamma]$}
\RightLabel{$I_8-U_i$}
\TrinaryInfC{$\stackrel{\wedge_i}{Id}(b,c,d)~\in~U_i~[\Gamma]$}
\DisplayProof

\vspace{0.2in}
\AxiomC{$a~\in~U_i~[\Gamma]$}
\RightLabel{$E-U_i$}
\UnaryInfC{$T_i(a)~type~[\Gamma]$}
\DisplayProof

\vspace{0.2in}
\AxiomC{$\Gamma~cont~$}
\RightLabel{$C_1-U_i$}
\UnaryInfC{$T_i(\stackrel{\wedge_i}{N_0})=N_0~type~[\Gamma]$}
\DisplayProof\qquad
\AxiomC{$\Gamma~cont~$}
\RightLabel{$C_2-U_i$}
\UnaryInfC{$T_i(\stackrel{\wedge_i}{N_1})=N_1~type~[\Gamma]$}
\DisplayProof\qquad
\AxiomC{$\Gamma~cont~$}
\RightLabel{$C_3-U_i$}
\UnaryInfC{$T_i(\stackrel{\wedge_i}{Nat})=Nat~type~[\Gamma]$}
\DisplayProof

\vspace{0.2in}
\AxiomC{$c(x)~\in~U_i~[\Gamma,x\in T(b)]$}
\AxiomC{$b~\in~U_i~[\Gamma]$}
\RightLabel{$C_4-U_i$}
\BinaryInfC{$T_i(\stackrel{\wedge_i}{\Sigma}_{x\in T(b)}c(x))=\Sigma_{x\in T(b)}T(c)~type~[\Gamma]$}
\DisplayProof\qquad
\AxiomC{$c(x)~\in~U_i~[\Gamma,x\in T(b)]$}
\AxiomC{$b~\in~U_i~[\Gamma]$}
\RightLabel{$C_5-U_i$}
\BinaryInfC{$T_i(\stackrel{\wedge_i}{\Pi}_{x\in T(b)}c(x))=\Pi{x\in T(b)}T(c)~type~[\Gamma]$}
\DisplayProof

\vspace{0.2in}
\AxiomC{$b~\in~U_i~[\Gamma]$}
\AxiomC{$c~\in~U_i~[\Gamma]$}
\RightLabel{$C_6-U_i$}
\BinaryInfC{$T_i(b\stackrel{\wedge_i}{+}c)=T(b)+T(c)~type~[\Gamma]$}
\DisplayProof\qquad
\AxiomC{$b~\in~U_i~[\Gamma]$}
\RightLabel{$C_7-U_i$}
\UnaryInfC{$T_i(\stackrel{\wedge_i}{List}(b))=List(T(b))~type~[\Gamma]$}
\DisplayProof

\vspace{0.2in}
\AxiomC{$b~\in~U_i~[\Gamma]$}
\AxiomC{$c~\in~T(b)~[\Gamma]$}
\AxiomC{$d~\in~T(b)~[\Gamma]$}
\RightLabel{$C_8-U_i$}
\TrinaryInfC{$T_i(\stackrel{\wedge_i}{Id}(b,c,d))=Id(T(b),c,d)~type~[\Gamma]$}
\DisplayProof

\vspace{0.2in}
Regole per la gerarchia di universi:

\vspace{0.2in}
\AxiomC{$\Gamma~cont$}
\RightLabel{$I_9-U_i$}
\UnaryInfC{$\stackrel{\wedge_i}{U_i}~\in~U_{i+1}~[\Gamma]$}
\DisplayProof\qquad
\AxiomC{$\Gamma~cont$}
\RightLabel{$C_9-U_i$}
\UnaryInfC{$T_{i+1}(\stackrel{\wedge_i}{U_i})=U_i~type~[\Gamma]$}
\DisplayProof

\vspace{0.2in}
Regole per l'operatore di innalzamento:

\vspace{0.2in}
\AxiomC{$\Gamma~cont$}
\RightLabel{$C_1-U_{i+1}$}
\UnaryInfC{$t_{i+1}(\stackrel{\wedge_i}{N_0})=\stackrel{\wedge_{i+1}}{N_0}~\in~U_{i+1}~[\Gamma]$}
\DisplayProof\qquad
\AxiomC{$\Gamma~cont$}
\RightLabel{$C_2-U_{i+1}$}
\UnaryInfC{$t_{i+1}(\stackrel{\wedge_i}{N_1})=\stackrel{\wedge_{i+1}}{N_1}~\in~U_{i+1}~[\Gamma]$}
\DisplayProof

\vspace{0.2in}
\AxiomC{$\Gamma~cont$}
\RightLabel{$C_3-U_{i+1}$}
\UnaryInfC{$t_{i+1}(\stackrel{\wedge_i}{Nat})=\stackrel{\wedge_{i+1}}{Nat}~\in~U_{i+1}~[\Gamma]$}
\DisplayProof\qquad
\AxiomC{$c(x)~\in~U_i~[\Gamma,x\in T(b)]$}
\AxiomC{$b~\in~U_i~[\Gamma]$}
\RightLabel{$C_4-U_{i+1}$}
\BinaryInfC{$t_{i+1}(\stackrel{\wedge_i}{\Sigma}_{x\in T(b)}c(x))=\stackrel{\wedge_{i+1}}{\Sigma}_{x\in T(b)}c(x)~\in~U_{i+1}~[\Gamma]$}
\DisplayProof

\vspace{0.2in}
\AxiomC{$c(x)~\in~U_i~[\Gamma,x\in T(b)]$}
\AxiomC{$b~\in~U_i~[\Gamma]$}
\RightLabel{$C_5-U_{i+1}$}
\BinaryInfC{$t_{i+1}(\stackrel{\wedge_i}{\Pi}_{x\in T(b)}c(x))=\stackrel{\wedge_{i+1}}{\Pi}_{x\in T(b)}c(x)~\in~U_{i+1}~[\Gamma]$}
\DisplayProof\qquad
\AxiomC{$b~\in~U_i~[\Gamma]$}
\AxiomC{$c~\in~U_i~[\Gamma]$}
\RightLabel{$C_6-U_{i+1}$}
\BinaryInfC{$t_{i+1}(b\stackrel{\wedge_i}{+}c)=b\stackrel{\wedge_{i+1}}{+}c~\in~U_{i+1}~[\Gamma]$}
\DisplayProof

\vspace{0.2in}
\AxiomC{$b~\in~U_i~[\Gamma]$}
\RightLabel{$C_7-U_{i+1}$}
\UnaryInfC{$t_{i+1}(\stackrel{\wedge_i}{List}(b))=\stackrel{\wedge_{i+1}}{List}(b)~\in~U_{i+1}~[\Gamma]$}
\DisplayProof

\vspace{0.2in}
\AxiomC{$b~\in~U_i~[\Gamma]$}
\AxiomC{$c~\in~T(b)~[\Gamma]$}
\AxiomC{$d~\in~T(b)~[\Gamma]$}
\RightLabel{$C_8-U_{i+1}$}
\TrinaryInfC{$t_{i+1}(\stackrel{\wedge_i}{Id}(b,c,d))=\stackrel{\wedge_{i+1}}{Id}(b,c,d)~\in~U_{i+1}~[\Gamma]$}
\DisplayProof

\paragraph{Russel} \mbox{} \\
Regole ``classiche'' modificate per tenere conto di infiniti universi:

\vspace{0.2in}
\AxiomC{$\Gamma~cont$}
\RightLabel{$F-U_i$}
\UnaryInfC{$U_i~type~[\Gamma]$}
\DisplayProof\qquad
\AxiomC{$\Gamma~cont$}
\RightLabel{$I_1-U_0$}
\UnaryInfC{$N_0~\in~U_0[\Gamma]$}
\DisplayProof\qquad
\AxiomC{$\Gamma~cont$}
\RightLabel{$I_2-U_0$}
\UnaryInfC{$N_1~\in~U_0[\Gamma]$}
\DisplayProof\qquad
\AxiomC{$\Gamma~cont$}
\RightLabel{$I_3-U_0$}
\UnaryInfC{$Nat~\in~U_0[\Gamma]$}
\DisplayProof\qquad

\vspace{0.2in}
\AxiomC{$C(x)\in U_i~[\Gamma,x\in B]$}
\AxiomC{$B\in U_i~[\Gamma]$}
\RightLabel{$I_4-U_i$}
\BinaryInfC{$\Sigma_{x\in B}C(x)\in U_i~[\Gamma]$}
\DisplayProof\qquad
\AxiomC{$C(x)\in U_i~[\Gamma,x\in B]$}
\AxiomC{$B\in U_i~[\Gamma]$}
\RightLabel{$I_5-U_i$}
\BinaryInfC{$\Pi_{x\in B}C(x)\in U_i~[\Gamma]$}
\DisplayProof

\vspace{0.2in}
\AxiomC{$B\in U_i~[\Gamma]$}
\AxiomC{$C\in U_i~[\Gamma]$}
\RightLabel{$I_6-U_i$}
\BinaryInfC{$B+C\in U_i~[\Gamma]$}
\DisplayProof\qquad
\AxiomC{$B\in U_i~[\Gamma]$}
\RightLabel{$I_7-U_i$}
\UnaryInfC{$List(B)\in U_i~[\Gamma]$}
\DisplayProof

\vspace{0.2in}
\AxiomC{$B\in U_i~[\Gamma]$}
\AxiomC{$c\in B~[\Gamma]$}
\AxiomC{$d\in B~[\Gamma]$}
\RightLabel{$I_7-U_i$}
\TrinaryInfC{$Id(B,c,d)\in U_i~[\Gamma]$}
\DisplayProof\qquad
\AxiomC{$A\in U_i~[\Gamma]$}
\RightLabel{$E-U_i$}
\UnaryInfC{$A~type~[\Gamma]$}
\DisplayProof

\vspace{0.2in}
Regole per la gerarchia di universi:

\vspace{0.2in}
\AxiomC{$U_i~type~[\Gamma]$}
\RightLabel{$I_9-U_{i+1}$}
\UnaryInfC{$U_i\in U_{i+1}~[\Gamma]$}
\DisplayProof

\vspace{0.2in}
Regole per l'inclusione tra universi:

\vspace{0.2in}
\AxiomC{$A\in U_i~[\Gamma]$}
\RightLabel{$I_9-U_{i+1}$}
\UnaryInfC{$A\in U_{i+1}~[\Gamma]$}
\DisplayProof

\subsection{Esercizio 2}
\begin{thm}
	Sia $\tau_t$ la teoria dei tipi con tutti i tipi finora introdotti, ma con il solo universo dei tipi piccoli à la Tarski. Sia $\tau_r$ la teoria dei tipi con tutti i tipi finora introdotti, ma con il solo universo dei tipi piccoli à la Russell. Dimostrare che $\tau_t$ e $\tau_r$ sono equivalenti.
\end{thm}
\paragraph{Parte 1 - Ottenere Russell da Tarski}
$\tau_t$ si basa sul fatto che $U_0$ contiene codifiche dei tipi, e non i tipi veri e propri. Al contrario, gli universi à la Russell non utilizzano nessuna codifica. Inoltre, le teorie $\tau_t$ e $\tau_r$ tengono conto solo dei tipi piccoli, e non ci sono regole che inseriscono $U_0$in $U_0$. Tenendo conto solo dei tipi codificati, per mostrare che da $\tau_t$ si può ottenere $\tau_r$, è sufficiente notare che, se un giudizio è derivabile codificando i tipi in $U_0$ allora lo è anche non codificandoli. Nessuno dei tipi inseriti in $U_0$ causa inconsistenza se non codificato. Infatti possiamo vedere una tipica regola à la Russell ($Nat\in U_0$), come un'applicazione di codifica e decodifica delle funzioni definite per Tarski: $T(\stackrel{\wedge}{Nat})\in U_0$, tenendo conto del fatto che, come si evince dalle regole di Tarski, dato $A~type$ si ha $T(\stackrel{\wedge}{A}~\equiv~A)$. In questo modo, per ogni giudizio derivabile in $\tau_t$, per ottenere il corrispondente giudizio derivabile in $\tau_r$ è sufficiente decodificare ogni codifica, ottenendo così il tipo originale e rendendo ammissibili le regole à la Russell.

\paragraph{Parte 2 - Ottenere Tarski da Russell}
$\tau_r$ si basa sul fatto che $U_0$ contiene direttamente i tipi, e non le loro codifiche. Di conseguenza è possibile ottenere il secondo dal primo dando le seguenti definizioni per la codifica ($\wedge$) e decodifica (T), dato $A~type~[\Gamma]$:
\[\stackrel{\wedge}{A}~\equiv~A\] e 
\[T(A)~\equiv~A\]
Cioè di fatto rendendo la codifica di un tipo $A$ esattamente uguale al tipo stesso. Quindi, un giudizio derivabile in Russell resta derivabile in Tarski.