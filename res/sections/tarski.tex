\section{Il tipo universo ``predicativo'' à la Tarski}
\subsection{Esercizio 1}
\begin{thm}
	Dare la definizione di eliminazione induttiva dell'universo $U_0$.
\end{thm}
Premesse:
\[a~\in~U_0~[\Gamma]\]
\[D(z)~type~[\Gamma,~z~\in~U_0]\]
\[a_1~\in~D(\stackrel{\wedge}{N_0})~[\Gamma]\]
\[a_2~\in~D(\stackrel{\wedge}{N_1})~[\Gamma]\]
\[a_3~\in~D(\stackrel{\wedge}{Nat})~[\Gamma]\]
\[a_4(x,y)~\in~D(\stackrel{\wedge}{List(x)})~[\Gamma,x~in~U_0,y\in D(x)]\]
\[a_5(x,y,z,u)~\in~D(\stackrel{\wedge}{Id}(x,y,z))~[\Gamma,x~\in~U_0,y\in T(x),z\in T(x), u\in D(x)]\]
\[a_6(x,y,z,u)~\in~D(x\stackrel{\wedge}{+}y)~[\Gamma,x\in U_0,y\in U_0,z\in D(x), u\in D(y)]\]
\[a_7(x,y,z,u)~\in~D(\stackrel{\wedge}{\Pi}_{w~\in~T(x)}y(w)))~[\Gamma,x\in U_0,y\in~T(x)~\to~U_0,z\in D(x), u\in D(c(x))]\]
\[a_8(x,y,z,u)~\in~D(\stackrel{\wedge}{\Sigma}_{w~\in~T(x)}y(w)))~[\Gamma,x\in U_0,y\in~T(x)~\to~U_0,z\in D(x), u\in D(c(x))]\]
\noindent\makebox[\linewidth]{\rule{\paperwidth}{0.4pt}}
\[El(a,a_1,a_2,a_3,a_4,a_5,a_6,a_7,a_8)~\in~D(a)\]

\subsection{Esercizio 2}
\begin{thm}
	Dimostrare che nella teoria dei tipi con tutti i costrutti introdotti \textit{senza il tipo universo $U_0$} si possono interpretare i tipi o nell'insieme vuoto $N_0~type~[\Gamma]$ o nell'insieme singoletto $N_1~type~[\Gamma]$ e i termini come elementi dell'interpretazione in modo tale che vengano soddisfatte le definizioni a pagina 29.
	
	Dimostrare poi che questa interpretazione \textit{rende vero} $Id(Nat,~0~1)~[]$ e quindi \textit{rende falso} $\neg Id(Nat,~0~1)~[]$. Ovvero l'interpretazione costruita è un \textit{modello in teoria dei tipi della proposizione} $0~=_{Nat}~1$ ed è un \textit{contromodello in teoria dei tipi della proposizione} $\neg 0~=_{Nat}~1$.
\end{thm}
\lstinputlisting[language=Coq]{res/code/tarski/es2.v}

\subsection{Esercizio 3}
\begin{thm}
	Dimostrare in teoria dei tipi con un universo, ad esempio $U_0$, che si può derivare
	\[\not~Id(Nat, 0, 1)~true~[]\]
\end{thm}
Viene definita una funzione $k:~Nat~\to~U_0$ tale che $k(0)~=~N_0$ e $k(1)~=~N_1$. Dopodiché si dimostra che vale $T(x)~\iff~T(y)~[\Gamma,~x~\in~U_0,~y~\in~U_0,~w~\in~Id(U_0,~x,~y)]$. Se è possibile dimostrare questo allora, ponendo $x~=~k(0)$ e $y~=~k(1)$ si ottiene una prova $w$ di $Id(U_0,~N_0,~N_1)$ e di conseguenza $T(x)~\iff~T(y)~\equiv~T(N_0)~\iff~T(N_1)~\equiv~N_0~\iff~N_1~\equiv~False$. È possibile definire $k$ e provare l'equivalenza come segue.
\lstinputlisting[language=Coq]{res/code/tarski/es3.v}